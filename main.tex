% The Ultimate Quest - LaTeX Book
\documentclass[12pt,a4paper]{book}

% Essential packages
\usepackage[utf8]{inputenc}
\usepackage{graphicx}
\usepackage{xcolor}
\usepackage{geometry}
\usepackage{fancyhdr}
\usepackage{titlesec}
\usepackage{enumitem}
\usepackage{amsmath,amssymb}
\usepackage{multicol}
\usepackage{etoolbox}
\usepackage{lipsum} % For demo text

% Page geometry - tighter margins for more content per page
\geometry{margin=0.8in}

% Color definitions
\definecolor{questblue}{RGB}{0,72,153}
\definecolor{questorange}{RGB}{255,127,0}

% Header and footer styling - more compact
\pagestyle{fancy}
\fancyhf{}
\fancyhead[LO,RE]{The Ultimate Quest}
\fancyhead[RO,LE]{\thepage}
\fancyfoot[C]{\textit{Engage | Explore | Excel}}
\renewcommand{\headrulewidth}{0.4pt}
\renewcommand{\footrulewidth}{0.4pt}
\setlength{\headheight}{13.6pt}

% Custom chapter and section styling - more compact
\titleformat{\chapter}[display]
{\normalfont\huge\bfseries\color{questblue}}
{\chaptertitlename\ \thechapter}{10pt}{\Huge}
\titlespacing*{\chapter}{0pt}{30pt}{20pt}

\titleformat{\section}
{\normalfont\Large\bfseries\color{questblue}}
{\thesection}{1em}{}
\titlespacing*{\section}{0pt}{2.5ex plus 1ex minus .2ex}{1.3ex plus .2ex}

% Global paragraph spacing reduction
\setlength{\parskip}{0.5em}
\setlength{\parindent}{0em}

% Adjust float placement to be less strict
\renewcommand{\topfraction}{0.9}
\renewcommand{\bottomfraction}{0.8}
\renewcommand{\textfraction}{0.07}
\renewcommand{\floatpagefraction}{0.7}

% MCQ counter setup
\newcounter{questioncounter}[section]
\newcounter{totalcounter}
\setcounter{totalcounter}{1} % Starting from 1

% Choose MCQ style: 
% For squares, uncomment the next line:
\input{mcq_style}
% For circles, uncomment the next line and comment the one above:
%\input{mcq_style_alt}
% For arrows, uncomment the next line and comment the two above:
%\input{mcq_style_arrow}

% Section Cover Page Template
\newcommand{\sectioncover}[3]{%
  \cleardoublepage
  \thispagestyle{empty}
  \begin{center}
    \vspace*{3cm}
    {\huge\bfseries\textcolor{questblue}{#1}\par}
    \vspace{1cm}
    {\Large\textit{#2}\par}
    \vspace{2cm}
    \begin{tcolorbox}[
      enhanced,
      colback=white,
      colframe=questorange,
      arc=5mm,
      boxrule=0.5mm,
      width=0.7\textwidth,
      halign=center,
      valign=center,
      height=6cm
    ]
      \begin{center}
        {\large #3\par}
      \end{center}
    \end{tcolorbox}
    \vspace{2cm}
    % Uncomment and modify to include an icon/image for the section
    % \includegraphics[width=0.2\textwidth]{#1_icon.png}
  \end{center}
  \cleardoublepage
} 

% Adjust section cover spacing
\makeatletter
\renewcommand{\sectioncover}[3]{%
  \cleardoublepage
  \thispagestyle{empty}
  \begin{center}
    \vspace*{2cm}
    {\huge\bfseries\textcolor{questblue}{#1}\par}
    \vspace{0.7cm}
    {\Large\textit{#2}\par}
    \vspace{1.2cm}
    \begin{tcolorbox}[
      enhanced,
      colback=white,
      colframe=questorange,
      arc=5mm,
      boxrule=0.5mm,
      width=0.7\textwidth,
      halign=center,
      valign=center,
      height=5cm
    ]
      \begin{center}
        {\large #3\par}
      \end{center}
    \end{tcolorbox}
    \vspace{1.5cm}
  \end{center}
  \cleardoublepage
}
\makeatother

\begin{document}

\frontmatter
\begin{titlepage}
    \centering
    
    \vspace*{1.5cm}
    
    {\huge\bfseries\textcolor{questorange}{LATEST EDITION}\par}
    \vspace{0.7cm}
    
    {\Huge\bfseries\textcolor{questblue}{THE ULTIMATE}\par}
    \vspace{0.3cm}
    
    {\Huge\bfseries\textcolor{questblue}{QUEST}\par}
    \vspace{0.2cm}
    
    {\Large\textit{South Asia's Largest Quiz}\par}
    \vspace{1cm}
    
    {\large A Power-Packed Quiz Book Covering\par}
    \vspace{0.3cm}
    
    {\Large\bfseries Science, Technology, Engineering, Arts \& Mathematics (STEAM)!\par}
    \vspace{1.5cm}
    
    {\Large\textit{\textcolor{questorange}{Engage | Explore | Excel}}\par}
    \vspace{1.5cm}
    
    {\Large\bfseries Gaurava Yadav\par}
    
    \vfill
    
    % Optional: add a logo or image here
    % \includegraphics[width=0.3\textwidth]{logo.png}
    
    \vspace{1.5cm}
\end{titlepage} 
\tableofcontents

\mainmatter

\chapter{Introduction to The Ultimate Quest}
\section{About This Book}
This book is designed to challenge your knowledge across Science, Technology, Engineering, Arts \& Mathematics (STEAM). Each section contains multiple-choice questions to test your understanding and expand your horizons.

\lipsum[1]

\sectioncover{SCIENCE}{Test your knowledge of the natural world}{Explore the wonders of Physics, Chemistry, Biology, and more as you tackle challenging questions about our universe and how it works.}

\chapter{Science}
\section{Physics}

\begin{enhancedmcq}[Fundamental forces]{Which of the following is a fundamental force in nature?}
    \incorrectoption{Magnetic force}
    \incorrectoption{Nuclear force}
    \correctoption{Gravitational force}
    \incorrectoption{Centrifugal force}
\end{enhancedmcq}

\begin{enhancedmcq}[SI Units]{What is the SI unit of electric current?}
    \incorrectoption{Volt}
    \correctoption{Ampere}
    \incorrectoption{Ohm}
    \incorrectoption{Coulomb}
\end{enhancedmcq}

\begin{enhancedmcq}[Famous Scientists]{Which scientist proposed the theory of relativity?}
    \incorrectoption{Isaac Newton}
    \incorrectoption{Niels Bohr}
    \correctoption{Albert Einstein}
    \incorrectoption{Marie Curie}
\end{enhancedmcq}

\begin{enhancedmcq}[Light]{What is the approximate speed of light in a vacuum?}
    \incorrectoption{300,000 km/h}
    \incorrectoption{3,000 km/s}
    \correctoption{300,000 km/s}
    \incorrectoption{30,000 km/s}
\end{enhancedmcq}

\begin{enhancedmcq}[Electromagnetism]{Which scientist discovered electromagnetic induction?}
    \incorrectoption{Thomas Edison}
    \correctoption{Michael Faraday}
    \incorrectoption{Nikola Tesla}
    \incorrectoption{James Clerk Maxwell}
\end{enhancedmcq}

\section{Chemistry}

\begin{enhancedmcq}[Chemical Elements]{What is the chemical symbol for gold?}
    \incorrectoption{Go}
    \incorrectoption{Gd}
    \correctoption{Au}
    \incorrectoption{Ag}
\end{enhancedmcq}

\begin{enhancedmcq}[Periodic Table]{Which of the following is a noble gas?}
    \incorrectoption{Nitrogen}
    \incorrectoption{Oxygen}
    \correctoption{Helium}
    \incorrectoption{Hydrogen}
\end{enhancedmcq}

\begin{enhancedmcq}[Acid-Base Chemistry]{What is the pH of a neutral solution?}
    \incorrectoption{0}
    \correctoption{7}
    \incorrectoption{14}
    \incorrectoption{1}
\end{enhancedmcq}

\begin{enhancedmcq}[Chemical Bonds]{Which type of bond involves the sharing of electrons?}
    \incorrectoption{Ionic bond}
    \correctoption{Covalent bond}
    \incorrectoption{Metallic bond}
    \incorrectoption{Hydrogen bond}
\end{enhancedmcq}

\begin{enhancedmcq}[Organic Chemistry]{What is the general formula for alkanes?}
    \incorrectoption{C$_n$H$_{2n}$}
    \correctoption{C$_n$H$_{2n+2}$}
    \incorrectoption{C$_n$H$_{2n-2}$}
    \incorrectoption{C$_n$H$_n$}
\end{enhancedmcq}

\sectioncover{TECHNOLOGY}{Explore the digital revolution}{From programming languages to computer architecture, test your knowledge of the technologies that power our modern world.}

\chapter{Technology}
\section{Computer Science}

\begin{enhancedmcq}[Programming Languages]{Which of the following is a high-level programming language?}
    \incorrectoption{Assembly}
    \incorrectoption{Machine code}
    \correctoption{Python}
    \incorrectoption{Binary}
\end{enhancedmcq}

\begin{enhancedmcq}[Computer Hardware]{What does CPU stand for?}
    \correctoption{Central Processing Unit}
    \incorrectoption{Computer Personal Unit}
    \incorrectoption{Central Program Utility}
    \incorrectoption{Control Processing Unit}
\end{enhancedmcq}

\begin{enhancedmcq}[Computing History]{Who is considered the father of computer science?}
    \incorrectoption{Bill Gates}
    \incorrectoption{Steve Jobs}
    \correctoption{Alan Turing}
    \incorrectoption{Mark Zuckerberg}
\end{enhancedmcq}

\begin{enhancedmcq}[Networking]{What does HTTP stand for?}
    \incorrectoption{High Transfer Text Protocol}
    \correctoption{Hypertext Transfer Protocol}
    \incorrectoption{Hypertext Technical Processing}
    \incorrectoption{High Technical Transfer Protocol}
\end{enhancedmcq}

\begin{enhancedmcq}[Operating Systems]{Which of the following is NOT an operating system?}
    \incorrectoption{Linux}
    \incorrectoption{Windows}
    \incorrectoption{macOS}
    \correctoption{Oracle}
\end{enhancedmcq}

\sectioncover{ENGINEERING}{Building our world}{Discover the principles and innovations behind mechanical, electrical, civil, and chemical engineering that shape our infrastructure and daily lives.}

\chapter{Engineering}
\section{Mechanical Engineering}

\begin{enhancedmcq}[Fluid Mechanics]{What is the principle that describes the behavior of an ideal fluid flowing through a pipe?}
    \incorrectoption{Newton's Law}
    \correctoption{Bernoulli's Principle}
    \incorrectoption{Archimedes' Principle}
    \incorrectoption{Hook's Law}
\end{enhancedmcq}

\begin{enhancedmcq}[Simple Machines]{Which of the following is NOT a type of simple machine?}
    \incorrectoption{Lever}
    \incorrectoption{Pulley}
    \incorrectoption{Wedge}
    \correctoption{Battery}
\end{enhancedmcq}

\begin{enhancedmcq}[Thermodynamics]{Which law of thermodynamics states that energy cannot be created or destroyed in an isolated system?}
    \correctoption{First law}
    \incorrectoption{Second law}
    \incorrectoption{Third law}
    \incorrectoption{Zeroth law}
\end{enhancedmcq}

\begin{enhancedmcq}[Materials Science]{Which material property describes its ability to withstand deformation under load?}
    \incorrectoption{Ductility}
    \incorrectoption{Malleability}
    \correctoption{Strength}
    \incorrectoption{Toughness}
\end{enhancedmcq}

\begin{enhancedmcq}[Engineering Design]{Which of the following is NOT a typical stage in the engineering design process?}
    \incorrectoption{Problem definition}
    \incorrectoption{Concept generation}
    \incorrectoption{Prototyping}
    \correctoption{Marketing}
\end{enhancedmcq}

\sectioncover{ARTS}{Celebrating human creativity}{From visual arts to literature, music, and history, explore the expressions of human creativity and cultural heritage.}

\chapter{Arts}
\section{History of Art}

\begin{enhancedmcq}[Famous Paintings]{Who painted the Mona Lisa?}
    \incorrectoption{Michelangelo}
    \correctoption{Leonardo da Vinci}
    \incorrectoption{Pablo Picasso}
    \incorrectoption{Vincent van Gogh}
\end{enhancedmcq}

\begin{enhancedmcq}[Art Movements]{Which art movement is characterized by dreamlike and incongruous images?}
    \incorrectoption{Impressionism}
    \incorrectoption{Cubism}
    \correctoption{Surrealism}
    \incorrectoption{Realism}
\end{enhancedmcq}

\begin{enhancedmcq}[Architecture]{Which architectural style features flying buttresses and pointed arches?}
    \incorrectoption{Classical}
    \incorrectoption{Baroque}
    \correctoption{Gothic}
    \incorrectoption{Art Deco}
\end{enhancedmcq}

\begin{enhancedmcq}[Literature]{Who wrote "Pride and Prejudice"?}
    \incorrectoption{Charlotte Brontë}
    \incorrectoption{Emily Brontë}
    \incorrectoption{Virginia Woolf}
    \correctoption{Jane Austen}
\end{enhancedmcq}

\begin{enhancedmcq}[Music]{Which of the following composers was deaf when he composed his Ninth Symphony?}
    \incorrectoption{Wolfgang Amadeus Mozart}
    \correctoption{Ludwig van Beethoven}
    \incorrectoption{Johann Sebastian Bach}
    \incorrectoption{Franz Joseph Haydn}
\end{enhancedmcq}

\sectioncover{MATHEMATICS}{The language of the universe}{Challenge yourself with questions about numbers, patterns, shapes, and the mathematical principles that underpin all sciences.}

\chapter{Mathematics}
\section{Algebra}

\begin{enhancedmcq}[Linear Equations]{What is the value of $x$ in the equation $2x + 5 = 15$?}
    \correctoption{5}
    \incorrectoption{10}
    \incorrectoption{7.5}
    \incorrectoption{5.5}
\end{enhancedmcq}

\begin{enhancedmcq}[Quadratic Equations]{Which of the following is NOT a solution to $x^2 - 4 = 0$?}
    \incorrectoption{2}
    \incorrectoption{-2}
    \correctoption{0}
    \incorrectoption{$\sqrt{4}$}
\end{enhancedmcq}

\begin{enhancedmcq}[Functions]{What is the range of the function $f(x) = \sqrt{x}$?}
    \incorrectoption{All real numbers}
    \incorrectoption{All positive real numbers}
    \correctoption{All non-negative real numbers}
    \incorrectoption{All negative real numbers}
\end{enhancedmcq}

\begin{enhancedmcq}[Sequences]{What is the next number in the Fibonacci sequence: 0, 1, 1, 2, 3, 5, 8, ?}
    \incorrectoption{12}
    \incorrectoption{11}
    \correctoption{13}
    \incorrectoption{15}
\end{enhancedmcq}

\section{Geometry}

\begin{enhancedmcq}[Triangles]{What is the sum of angles in a triangle?}
    \incorrectoption{$90^\circ$}
    \correctoption{$180^\circ$}
    \incorrectoption{$270^\circ$}
    \incorrectoption{$360^\circ$}
\end{enhancedmcq}

\begin{enhancedmcq}[Circles]{What is the formula for the area of a circle?}
    \incorrectoption{$\pi r$}
    \incorrectoption{$2\pi r$}
    \correctoption{$\pi r^2$}
    \incorrectoption{$2\pi r^2$}
\end{enhancedmcq}

\begin{enhancedmcq}[Polygons]{How many sides does a heptagon have?}
    \incorrectoption{5}
    \incorrectoption{6}
    \correctoption{7}
    \incorrectoption{8}
\end{enhancedmcq}

\begin{enhancedmcq}[Coordinate Geometry]{What is the distance between the points (0,0) and (3,4)?}
    \incorrectoption{7}
    \incorrectoption{3.5}
    \correctoption{5}
    \incorrectoption{6}
\end{enhancedmcq}

\begin{enhancedmcq}[Solid Geometry]{What is the volume of a cube with side length $s$?}
    \incorrectoption{$s^2$}
    \incorrectoption{$6s^2$}
    \correctoption{$s^3$}
    \incorrectoption{$4s^3$}
\end{enhancedmcq}

% Answer key section
\sectioncover{ANSWER KEY}{Check your knowledge}{Review your answers and track your progress through each section of The Ultimate Quest.}

\answerkey
\answer{c} % Fundamental forces
\answer{b} % SI Units
\answer{c} % Famous Scientists
\answer{c} % Light
\answer{b} % Electromagnetism
\answer{c} % Chemical Elements
\answer{c} % Periodic Table
\answer{b} % Acid-Base Chemistry
\answer{b} % Chemical Bonds
\answer{b} % Organic Chemistry
\answer{c} % Programming Languages
\answer{a} % Computer Hardware
\answer{c} % Computing History
\answer{b} % Networking
\answer{d} % Operating Systems
\answer{b} % Fluid Mechanics
\answer{d} % Simple Machines
\answer{a} % Thermodynamics
\answer{c} % Materials Science
\answer{d} % Engineering Design
\answer{b} % Famous Paintings
\answer{c} % Art Movements
\answer{c} % Architecture
\answer{d} % Literature
\answer{b} % Music
\answer{a} % Linear Equations
\answer{c} % Quadratic Equations
\answer{c} % Functions
\answer{c} % Sequences
\answer{b} % Triangles
\answer{c} % Circles
\answer{c} % Polygons
\answer{c} % Coordinate Geometry
\answer{c} % Solid Geometry
\endanswerkey

\end{document} 