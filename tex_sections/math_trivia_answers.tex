\section{ANSWER KEY: Math Trivia}

\textbf{Q1} b) Charles Babbage\par
\textbf{Q2} b) Shakuntala Devi\par
\textbf{Q3} c) Group Theory\par
\textbf{Q4} b) By calculating a Taylor‑series approximation error term\par
\textbf{Q5} a) Babylonians\par
\textbf{Q6} b) India and Mayan civilizations\par
\textbf{Q7} b) Euclid\par
\textbf{Q8} a) René Descartes\par
\textbf{Q9} a) Logarithms\par
\textbf{Q10} b) To avoid writing "is equal to" repeatedly\par
\textbf{Q11} b) Egyptians\par
\textbf{Q12} a) Cardano and Tartaglia\par
\textbf{Q13} a) Laws of planetary motion\par
\textbf{Q14} b) Gottfried Wilhelm Leibniz\par
\textbf{Q15} a) Designing furniture layouts\par
\textbf{Q16} b) 1557\par
\textbf{Q17} b) John Napier\par
\textbf{Q18} a) René Descartes\par
\textbf{Q19} b) Mayan\par
\textbf{Q20} a) Algebra\par
\textbf{Q21} b) Euclid's *Elements*\par
\textbf{Q22} b) Diophantus of Alexandria\par
\textbf{Q23} a) *Liber Abaci*\par
\textbf{Q24} b) Classical mechanics\par
\textbf{Q25} a) Carl Friedrich Gauss\par
\textbf{Q26} b) Cartesian coordinates\par
\textbf{Q27} b) Symbolic logic\par
\textbf{Q28} a) *Mathematische Grundlagen der Quantenmechanik*\par
\textbf{Q29} b) *Theory of Games and Economic Behavior*\par
\textbf{Q30} a) *Grundbegriffe der Wahrscheinlichkeitsrechnung*\par
\textbf{Q31} b) *On Formally Undecidable Propositions*\par
\textbf{Q32} a) *A History of Mathematics*\par
\textbf{Q33} b) William Dunham\par
\textbf{Q34} b) Fermat's Last Theorem\par
\textbf{Q35} b) Beginner‑friendly history of mathematics\par
\textbf{Q36} a) Calculus and its impact on science\par
\textbf{Q37} b) Lighthearted exploration of math in everyday life\par
\textbf{Q38} a) Applications of math in daily decision‑making\par
\textbf{Q39} b) Advanced students and professionals\par
\textbf{Q40} b) A fictional two‑dimensional world\par
\textbf{Q41} b) The rabbit problem\par
\textbf{Q42} a) *Flatland*\par
\textbf{Q43} b) Modern mathematics across disciplines\par
\textbf{Q44} a) Famous mathematical breakthroughs\par
\textbf{Q45} b) *Mathematische Grundlagen der Quantenmechanik*\par
\textbf{Q46} a) John von Neumann and Oskar Morgenstern\par
\textbf{Q47} b) Syncopated algebraic notation\par
\textbf{Q48} b) Formal mathematical systems\par
\textbf{Q49} b) A historical overview for beginners\par
\textbf{Q50} b) Calculus in understanding the universe\par