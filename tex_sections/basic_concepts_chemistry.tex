\section{BASIC CONCEPTS OF CHEMISTRY}

\textbf{Q1} What is the smallest unit of matter that retains the properties of an element, and why is it crucial in understanding chemical reactions?\par
\quad - a) Molecule\par
\quad - b) Compound\par
\quad - c) Atom\par

\textbf{Q2} Which law states that matter cannot be created or destroyed in a chemical reaction, and how does this impact our understanding of chemical processes?\par
\quad - a) Law of Conservation of Mass\par
\quad - b) Law of Definite Proportions\par
\quad - c) Law of Multiple Proportions\par

\textbf{Q3} What type of bond involves the sharing of electrons between atoms, and how does this differ from other types of chemical bonds?\par
\quad - a) Ionic Bond\par
\quad - b) Covalent Bond\par
\quad - c) Hydrogen Bond\par

\textbf{Q4} Which branch of chemistry deals with the study of organic compounds, and what are some of its key applications?\par
\quad - a) Inorganic Chemistry\par
\quad - b) Organic Chemistry\par
\quad - c) Physical Chemistry\par

\textbf{Q5} What is the empirical formula of a compound, and how does it provide insight into the composition of substances?\par
\quad - a) Molecular Formula\par
\quad - b) Structural Formula\par
\quad - c) Empirical Formula\par

\textbf{Q6} How do atoms gain or lose electrons to form ions, and what role does this play in chemical reactions?\par
\quad - a) Ionization\par
\quad - b) Electrolysis\par
\quad - c) Ion Formation\par

\textbf{Q7} Which type of reaction involves the transfer of electrons from one species to another, and what are some common examples of this process?\par
\quad - a) Acid‑Base Reaction\par
\quad - b) Redox Reaction\par
\quad - c) Synthesis Reaction\par

\textbf{Q8} What is the term for the energy required to remove an electron from an atom, and how does this relate to atomic properties?\par
\quad - a) Ionization Energy\par
\quad - b) Electron Affinity\par
\quad - c) Electronegativity\par

\textbf{Q9} Which principle explains why elements combine in simple whole‑number ratios, and how does this apply to chemical compounds?\par
\quad - a) Law of Conservation of Mass\par
\quad - b) Law of Definite Proportions\par
\quad - c) Law of Multiple Proportions\par

\textbf{Q10} What is the term for the study of the rates of chemical reactions, and how does this help us understand reaction mechanisms?\par
\quad - a) Chemical Kinetics\par
\quad - b) Chemical Thermodynamics\par
\quad - c) Chemical Equilibrium\par

\textbf{Q11} Which type of compound is formed when two or more different elements are chemically bonded, and what are some common examples?\par
\quad - a) Element\par
\quad - b) Compound\par
\quad - c) Mixture\par

\textbf{Q12} What is the process by which a solid changes directly to a gas, and how does this occur in everyday life?\par
\quad - a) Melting\par
\quad - b) Boiling\par
\quad - c) Sublimation\par

\textbf{Q13} Which type of bond involves the attraction between a hydrogen atom bonded to a highly electronegative atom and another electronegative atom?\par
\quad - a) Hydrogen Bond\par
\quad - b) Ionic Bond\par
\quad - c) Covalent Bond\par

\textbf{Q14} What is the term for the amount of heat required to raise the temperature of one gram of a substance by one degree Celsius?\par
\quad - a) Specific Heat Capacity\par
\quad - b) Latent Heat\par
\quad - c) Heat of Fusion\par

\textbf{Q15} Which branch of chemistry deals with the study of the physical properties and behavior of matter?\par
\quad - a) Physical Chemistry\par
\quad - b) Organic Chemistry\par
\quad - c) Inorganic Chemistry\par

\textbf{Q16} What is the term for the minimum amount of energy required for a reaction to occur, and how does this influence reaction rates?\par
\quad - a) Activation Energy\par
\quad - b) Reaction Energy\par
\quad - c) Catalyst Energy\par

\textbf{Q17} Which type of reaction involves the combination of two or more substances to form a new compound?\par
\quad - a) Synthesis Reaction\par
\quad - b) Decomposition Reaction\par
\quad - c) Replacement Reaction\par

\textbf{Q18} What is the process by which a liquid changes to a gas, and what factors affect this process?\par
\quad - a) Evaporation\par
\quad - b) Condensation\par
\quad - c) Boiling\par

\textbf{Q19} Which type of mixture has components that are not uniformly distributed?\par
\quad - a) Solution\par
\quad - b) Suspension\par
\quad - c) Colloid\par

\textbf{Q20} What is the term for the study of the equilibrium state of chemical reactions, and how does this help predict reaction outcomes?\par
\quad - a) Chemical Equilibrium\par
\quad - b) Chemical Kinetics\par
\quad - c) Chemical Thermodynamics\par

\textbf{Q21} Which type of compound is formed when two or more atoms of the same element are chemically bonded?\par
\quad - a) Element\par
\quad - b) Compound\par
\quad - c) Molecule\par

\textbf{Q22} What is the process by which a gas changes directly to a solid?\par
\quad - a) Deposition\par
\quad - b) Sublimation\par
\quad - c) Condensation\par

\textbf{Q23} Which type of bond involves the transfer of electrons from one atom to another?\par
\quad - a) Ionic Bond\par
\quad - b) Covalent Bond\par
\quad - c) Hydrogen Bond\par

\textbf{Q24} What is the term for the energy released or absorbed during a chemical reaction?\par
\quad - a) Heat of Reaction\par
\quad - b) Activation Energy\par
\quad - c) Reaction Energy\par

\textbf{Q25} Which branch of chemistry deals with the study of compounds not containing carbon?\par
\quad - a) Organic Chemistry\par
\quad - b) Inorganic Chemistry\par
\quad - c) Physical Chemistry\par

\textbf{Q26} What is the term for the ability of an atom to attract electrons in a covalent bond, and how does this influence molecular structure?\par
\quad - a) Electronegativity\par
\quad - b) Electropositivity\par
\quad - c) Ionization Energy\par

\textbf{Q27} Which type of reaction involves the breaking of a chemical bond using light energy, and what are some applications of this process?\par
\quad - a) Photolysis\par
\quad - b) Electrolysis\par
\quad - c) Catalysis\par

\textbf{Q28} What is the process by which a solid changes directly to a liquid, and how does this differ from other phase transitions?\par
\quad - a) Melting\par
\quad - b) Boiling\par
\quad - c) Sublimation\par

\textbf{Q29} Which type of mixture has components that are uniformly distributed, and what are some examples of this type of mixture?\par
\quad - a) Solution\par
\quad - b) Suspension\par
\quad - c) Colloid\par

\textbf{Q30} What is the term for the study of the behavior of gases, and how does this relate to the kinetic theory of gases?\par
\quad - a) Gas Laws\par
\quad - b) Liquid Laws\par
\quad - c) Solid Laws\par

\textbf{Q31} Which type of compound is formed when a metal reacts with a nonmetal, and what are some common examples of these compounds?\par
\quad - a) Ionic Compound\par
\quad - b) Covalent Compound\par
\quad - c) Molecular Compound\par

\textbf{Q32} What is the term for the energy required to break a chemical bond, and how does this relate to bond strength?\par
\quad - a) Bond Energy\par
\quad - b) Activation Energy\par
\quad - c) Reaction Energy\par

\textbf{Q33} Which type of reaction involves the replacement of one element by another in a compound?\par
\quad - a) Replacement Reaction\par
\quad - b) Synthesis Reaction\par
\quad - c) Decomposition Reaction\par

\textbf{Q34} What is the process by which a liquid changes to a solid, and what factors affect this process?\par
\quad - a) Freezing\par
\quad - b) Boiling\par
\quad - c) Condensation\par

\textbf{Q35} Which type of bond involves the sharing of electrons between two atoms of different electronegativities?\par
\quad - a) Polar Covalent Bond\par
\quad - b) Nonpolar Covalent Bond\par
\quad - c) Ionic Bond\par

\textbf{Q36} What is the term for the study of the structure of atoms and molecules, and how does this relate to quantum mechanics?\par
\quad - a) Atomic Structure\par
\quad - b) Molecular Structure\par
\quad - c) Quantum Mechanics\par

\textbf{Q37} Which type of compound is formed when two nonmetals react, and what are some common examples of these compounds?\par
\quad - a) Ionic Compound\par
\quad - b) Covalent Compound\par
\quad - c) Molecular Compound\par

\textbf{Q38} What is the term for the energy released when a chemical bond is formed, and how does this relate to bond formation?\par
\quad - a) Bond Energy\par
\quad - b) Activation Energy\par
\quad - c) Reaction Energy\par

\textbf{Q39} Which type of reaction involves the breaking down of a compound into simpler substances?\par
\quad - a) Decomposition Reaction\par
\quad - b) Synthesis Reaction\par
\quad - c) Replacement Reaction\par

\textbf{Q40} What is the process by which a gas changes to a liquid, and what factors affect this process?\par
\quad - a) Condensation\par
\quad - b) Evaporation\par
\quad - c) Boiling\par

\textbf{Q41} Which type of mixture has components that are not chemically bonded, and what are some examples of this type of mixture?\par
\quad - a) Solution\par
\quad - b) Suspension\par
\quad - c) Compound\par

\textbf{Q42} What is the term for the study of the properties of acids and bases, and how does this relate to chemical reactions?\par
\quad - a) Acid‑Base Chemistry\par
\quad - b) Redox Chemistry\par
\quad - c) Equilibrium Chemistry\par

\textbf{Q43} Which type of bond involves the attraction between two atoms of different electronegativities?\par
\quad - a) Polar Covalent Bond\par
\quad - b) Nonpolar Covalent Bond\par
\quad - c) Ionic Bond\par

\textbf{Q44} What is the term for the energy required to initiate a chemical reaction, and how does this influence reaction rates?\par
\quad - a) Activation Energy\par
\quad - b) Reaction Energy\par
\quad - c) Bond Energy\par

\textbf{Q45} Which type of reaction involves the transfer of electrons from one species to another?\par
\quad - a) Redox Reaction\par
\quad - b) Acid‑Base Reaction\par
\quad - c) Synthesis Reaction\par

\textbf{Q46} What is the process by which a solid changes directly to a gas without going through the liquid phase?\par
\quad - a) Sublimation\par
\quad - b) Deposition\par
\quad - c) Condensation\par

\textbf{Q47} Which type of compound is formed when a metal reacts with a polyatomic ion, and what are some examples of these compounds?\par
\quad - a) Ionic Compound\par
\quad - b) Covalent Compound\par
\quad - c) Molecular Compound\par

\textbf{Q48} What is the term for the study of the rates of chemical reactions, and how does this help us understand reaction mechanisms?\par
\quad - a) Chemical Kinetics\par
\quad - b) Chemical Thermodynamics\par
\quad - c) Chemical Equilibrium\par

\textbf{Q49} Which type of bond involves the sharing of electrons between two atoms of the same electronegativity?\par
\quad - a) Nonpolar Covalent Bond\par
\quad - b) Polar Covalent Bond\par
\quad - c) Ionic Bond\par

\textbf{Q50} What is the term for the energy released or absorbed during a phase change, and how does this relate to latent heat?\par
\quad - a) Latent Heat\par
\quad - b) Specific Heat\par
\quad - c) Heat of Reaction\par