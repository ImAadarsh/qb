
\begin{enhancedmcq}{When is Engineers' Day celebrated in India?}
\item September 13
\item September 14
\item September 15

\end{enhancedmcq}
\begin{enhancedmcq}{Engineers' Day in India celebrates the birth anniversary of which famous engineer?}
\item Mokshagundam Visvesvaraya
\item Jawaharlal Nehru
\item APJ Abdul Kalam

\end{enhancedmcq}
\begin{enhancedmcq}{Who is often referred to as the "Father of the Indian Space Program"?}
\item C. V. Raman
\item Vikram Sarabhai
\item Homi J. Bhabha

\end{enhancedmcq}
\begin{enhancedmcq}{Which Indian engineer is known for designing the iconic Qutub Minar in Delhi?}
\item Vishwakarma
\item Raja Raja Chola
\item Ustad Ahmad Lahori

\end{enhancedmcq}
\begin{enhancedmcq}{The famous Indian engineer Mokshagundam Visvesvaraya is best known for his contributions to which sector?}
\item Space Technology
\item Civil Engineering
\item Information Technology

\end{enhancedmcq}
\begin{enhancedmcq}{Who was the chief architect behind the design and construction of the Amritsar Golden Temple (Harmandir Sahib)?}
\item Le Corbusier
\item Guru Ram Das
\item Sir Edwin Lutyens

\end{enhancedmcq}
\begin{enhancedmcq}{Which Indian engineer is responsible for the design and construction of the Statue of Unity, the world's tallest statue?}
\item E. Sreedharan
\item A. P. J. Abdul Kalam
\item Sardar Patel

\end{enhancedmcq}
\begin{enhancedmcq}{Who is often called the "Missile Man of India" for his contributions to the Indian defence and space programs?}
\item Homi J. Bhabha
\item Satish Dhawan
\item A. P. J. Abdul Kalam

\end{enhancedmcq}
\begin{enhancedmcq}{Which Indian engineer pioneered the development of the indigenous light combat aircraft, Tejas?}
\item Raghuram Rajan
\item G. Madhavan Nair
\item Dr. Kota Harinarayana

\end{enhancedmcq}
\begin{enhancedmcq}{Who is known as the "Metro Man of India" for his work on the Delhi Metro and other metro projects?}
\item E. Sreedharan
\item Vikram Sarabhai
\item Mokshagundam Visvesvaraya

\end{enhancedmcq}
\begin{enhancedmcq}{What prestigious award was Mokshagundam Visvesvaraya honored with that is considered India's highest civilian award?}
\item Padma Vibhushan
\item Bharat Ratna
\item Padma Bhushan

\end{enhancedmcq}
\begin{enhancedmcq}{Which dam in India was built under the guidance of Sir Mokshagundam Visvesvaraya?}
\item Krishna Raja Sagara (KRS) Dam
\item Bhakra Nangal Dam
\item Tehri Dam

\end{enhancedmcq}
\begin{enhancedmcq}{Who developed India's first indigenous supercomputer, PARAM?}
\item Vijay Bhatkar
\item Sam Pitroda
\item F. C. Kohli

\end{enhancedmcq}
\begin{enhancedmcq}{Which Indian engineer is credited with developing the Pentium processor while working at Intel?}
\item Ajay Bhatt
\item Vinod Dham
\item Thomas Kurian

\end{enhancedmcq}
\begin{enhancedmcq}{Who is considered the "Father of Indian Industry"?}
\item Jamsetji Tata
\item G. D. Birla
\item Dhirubhai Ambani

\end{enhancedmcq}
\begin{enhancedmcq}{Which female Indian engineer is known as the "Missile Woman of India"?}
\item Tessy Thomas
\item Kalpana Chawla
\item Ritu Karidhal

\end{enhancedmcq}
\begin{enhancedmcq}{Which ancient Indian engineering marvel is famous for its corrosion‑resistant iron pillar?}
\item Iron Pillar of Delhi
\item Sun Temple of Konark
\item Brihadeshwara Temple

\end{enhancedmcq}
\begin{enhancedmcq}{Which Indian engineer led the team that designed and built India's first satellite, Aryabhata?}
\item U. R. Rao
\item Satish Dhawan
\item K. Kasturirangan

\end{enhancedmcq}
\begin{enhancedmcq}{Which ancient Indian civilization is known for its advanced urban planning and engineering?}
\item Indus Valley Civilization
\item Mauryan Empire
\item Gupta Empire

\end{enhancedmcq}
\begin{enhancedmcq}{Who is credited with developing India's civilian nuclear program?}
\item Homi J. Bhabha
\item Raja Ramanna
\item A. P. J. Abdul Kalam

\end{enhancedmcq}
\begin{enhancedmcq}{Which Indian engineer led the Mars Orbiter Mission (Mangalyaan) project?}
\item K. Radhakrishnan
\item Mylswamy Annadurai
\item K. Sivan

\end{enhancedmcq}
\begin{enhancedmcq}{India's first indigenous automotive engine, the Revtron, was developed by which company?}
\item Tata Motors
\item Mahindra & Mahindra
\item Maruti Suzuki

\end{enhancedmcq}
\begin{enhancedmcq}{Which ancient irrigation system from Tamil Nadu is still in use and is considered one of the oldest water management systems in the world?}
\item Grand Anicut (Kallanai)
\item Suranga System
\item Kunds of Rajasthan

\end{enhancedmcq}
\begin{enhancedmcq}{Who is known for establishing the Indian Institute of Science (IISc) in Bangalore?}
\item Jamsetji Tata
\item Mokshagundam Visvesvaraya
\item C. V. Raman

\end{enhancedmcq}
\begin{enhancedmcq}{Which Indian engineer is considered the pioneer of wireless communication?}
\item Jagadish Chandra Bose
\item G. N. Ramachandran
\item Satyendra Nath Bose

\end{enhancedmcq}
\begin{enhancedmcq}{Which Indian engineer contributed to the invention of fiber optics?}
\item Narinder Singh Kapany
\item C. K. N. Patel
\item Amar Bose

\end{enhancedmcq}
\begin{enhancedmcq}{Who led India's first nuclear test, Operation Smiling Buddha, in 1974?}
\item Raja Ramanna
\item Homi J. Bhabha
\item Vikram Sarabhai

\end{enhancedmcq}
\begin{enhancedmcq}{Which indigenous Indian combat helicopter is named after a venomous snake?}
\item Light Combat Helicopter
\item Advanced Light Helicopter
\item Dhruv

\end{enhancedmcq}
\begin{enhancedmcq}{Which dam in India is the highest gravity dam?}
\item Bhakra Nangal Dam
\item Tehri Dam
\item Sardar Sarovar Dam

\end{enhancedmcq}
\begin{enhancedmcq}{Who is known for developing India's Electronic Voting Machine (EVM) system?}
\item M. Natarajan
\item Sam Pitroda
\item T. N. Seshan

\end{enhancedmcq}
\begin{enhancedmcq}{Which Indian engineering achievement is the world's tallest railway bridge?}
\item Chenab Bridge
\item Pamban Bridge
\item Bogibeel Bridge

\end{enhancedmcq}
\begin{enhancedmcq}{Which Indian‑born engineer is the inventor of the USB (Universal Serial Bus)?}
\item Ajay Bhatt
\item Vinod Dham
\item Sabeer Bhatia

\end{enhancedmcq}
\begin{enhancedmcq}{Which structure, designed by an Indian engineer, has the world's largest single‑span stone dome?}
\item Gol Gumbaz in Bijapur
\item Bibi Ka Maqbara in Aurangabad
\item Taj Mahal in Agra

\end{enhancedmcq}
\begin{enhancedmcq}{Who established the first engineering college in India?}
\item Lord Dalhousie
\item James Thomason
\item Lord Macaulay

\end{enhancedmcq}
\begin{enhancedmcq}{Which Indian engineer is known for designing the structural framework of the Howrah Bridge?}
\item Rendel Palmer & Tritton
\item Sir Bradford Leslie
\item Sir Rajendra Nath Mookerjee

\end{enhancedmcq}
\begin{enhancedmcq}{Which indigenous Indian tactical surface‑to‑surface missile has a range of over 5000 km?}
\item Agni‑V
\item Prithvi
\item BrahMos

\end{enhancedmcq}
\begin{enhancedmcq}{Who developed India's first indigenous microprocessor, Shakti?}
\item IIT Madras researchers
\item DRDO scientists
\item ISRO engineers

\end{enhancedmcq}
\begin{enhancedmcq}{Which ancient engineering marvel in India features perfect acoustics where a sound made at a specific spot can be heard throughout the structure?}
\item Golconda Fort
\item Gol Gumbaz
\item Ranakpur Temple

\end{enhancedmcq}
\begin{enhancedmcq}{Which Indian transportation engineering project is the longest expressway in India?}
\item Yamuna Expressway
\item Mumbai‑Pune Expressway
\item Delhi‑Mumbai Expressway

\end{enhancedmcq}
\begin{enhancedmcq}{Which indigenous Indian naval ship was the first aircraft carrier built in India?}
\item INS Vikramaditya
\item INS Vikrant
\item INS Viraat

\end{enhancedmcq}
\begin{enhancedmcq}{Who established the first formal silk manufacturing industry in India?}
\item Tipu Sultan
\item Jamsetji Tata
\item G. D. Birla

\end{enhancedmcq}
\begin{enhancedmcq}{Which Indian scientist‑engineer is known for developing the first operational Indian parallel supercomputer?}
\item Vijay Bhatkar
\item Sam Pitroda
\item Faqir Chand Kohli

\end{enhancedmcq}
\begin{enhancedmcq}{Which ancient Indian mathematical text contains early engineering principles?}
\item Sulba Sutras
\item Aryabhatiya
\item Lilavati

\end{enhancedmcq}
\begin{enhancedmcq}{Which Indian structure is known for its earthquake‑resistant design despite being built in the 13th century?}
\item Sun Temple at Konark
\item Qutub Minar
\item Khajuraho Temples

\end{enhancedmcq}
\begin{enhancedmcq}{Which Indian spacecraft was the first to successfully enter Mars orbit on its first attempt?}
\item Chandrayaan‑1
\item Mangalyaan
\item Chandrayaan‑2

\end{enhancedmcq}
\begin{enhancedmcq}{Which Indian institute was established first with the focus on engineering education?}
\item IIT Kharagpur
\item IIT Bombay
\item IIT Madras

\end{enhancedmcq}
\begin{enhancedmcq}{Which Indian mathematician‑engineer is known for his contributions to algebra and trigonometry that later influenced engineering?}
\item Aryabhata
\item Brahmagupta
\item Bhaskara II

\end{enhancedmcq}
\begin{enhancedmcq}{Which indigenous Indian missile defense system was developed to intercept ballistic missiles?}
\item Advanced Air Defence (AAD)
\item Akash Missile System
\item Nag Missile

\end{enhancedmcq}
\begin{enhancedmcq}{Which ancient Indian university had extensive engineering studies as part of its curriculum?}
\item Nalanda University
\item Takshashila University
\item Vikramashila University

\end{enhancedmcq}
\begin{enhancedmcq}{Who designed the Indian Parliament House in New Delhi?}
\item Edwin Lutyens
\item Herbert Baker
\item Charles Correa
\end{enhancedmcq}