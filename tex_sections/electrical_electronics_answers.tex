\answerkey
\answer{b} Michael Faraday
\answer{c} Ohm
\answer{b} To step up or step down voltage levels
\answer{c} Ohm's Law
\answer{a} Pearl Street Station
\answer{b} To allow current to flow in one direction
\answer{b} Silicon
\answer{b} Electromagnetic induction
\answer{b} Michael Faraday
\answer{a} Coulomb
\answer{a} Light Emitting Diode
\answer{b} Direct Current (DC)
\answer{a} Lower power loss
\answer{a} Operating speed
\answer{b} Solar power
\answer{a} Root Mean Square
\answer{b} To store and release energy
\answer{a} They describe the relationship between electricity and magnetism
\answer{b} Ammeter
\answer{b} To automatically stop current flow during overloads
\answer{b} 60 Hz
\answer{a} Silicon
\answer{a} Nikola Tesla
\answer{a} To block DC and pass AC
\answer{a} Total Harmonic Distortion
\answer{b} Higher power delivery efficiency
\answer{a} Converting AC to DC
\answer{b} Transistor
\answer{b} To protect against electrical shock and equipment damage
\answer{a} Impedance
\answer{b} To protect against overcurrent conditions
\answer{c} Both P-type and N-type
\answer{a} Rectification
\answer{c} To switch on/off high currents
\answer{b} Dielectric heating
\answer{a} Electromagnetic Interference
\answer{b} To remove unwanted frequencies
\answer{c} Wattmeter
\answer{a} Efficiency
\answer{b} Copper
\answer{b} To protect against voltage spikes
\answer{a} Uninterruptible Power Supply
\answer{a} Electromagnetic induction
\answer{b} Resistance
\answer{b} To control high currents with low currents
\answer{a} Induction motor
\answer{b} To step up or step down voltage levels
\answer{a} Switch Mode Power Supply
\answer{b} Electromagnetic radiation
\answer{c} Solid-state physics
\endanswerkey