\section{MATH TRIVIA}

\textbf{Q1} Which mathematician is known as the "Father of the Computer" and was famous for hosting extravagant parties?\par
\quad - a) Alan Turing\par
\quad - b) Charles Babbage\par
\quad - c) John Napier\par

\textbf{Q2} Who was portrayed by Vidya Balan in the 2020 biopic, also known as the "Human Computer"?\par
\quad - a) Mahashweta Devi\par
\quad - b) Shakuntala Devi\par
\quad - c) Seema Sharma\par

\textbf{Q3} What mathematical idea did Évariste Galois work on the night before his duel?\par
\quad - a) Calculus\par
\quad - b) Pythagorean Theorem\par
\quad - c) Group Theory\par

\textbf{Q4} How did Igor Tamm prove he was a mathematician and save his life?\par
\quad - a) By solving a quadratic equation\par
\quad - b) By calculating a Taylor‑series approximation error term\par
\quad - c) By deriving the Pythagorean theorem\par

\textbf{Q5} Which ancient civilization approximated π as 3?\par
\quad - a) Babylonians\par
\quad - b) Greeks\par
\quad - c) Egyptians\par

\textbf{Q6} Where did the concept of zero as a number emerge independently?\par
\quad - a) Egypt and Greece\par
\quad - b) India and Mayan civilizations\par
\quad - c) Babylonia and Rome\par

\textbf{Q7} Who introduced axiomatic reasoning in mathematics?\par
\quad - a) Pythagoras\par
\quad - b) Euclid\par
\quad - c) Cardano\par

\textbf{Q8} Which mathematician invented Cartesian coordinates?\par
\quad - a) René Descartes\par
\quad - b) Isaac Newton\par
\quad - c) Johannes Kepler\par

\textbf{Q9} What mathematical discovery simplified complex calculations and was instrumental for astronomers like Kepler?\par
\quad - a) Logarithms\par
\quad - b) Calculus\par
\quad - c) Group Theory\par

\textbf{Q10} Why did Robert Recorde invent the equals sign (=)?\par
\quad - a) To simplify algebraic geometry\par
\quad - b) To avoid writing "is equal to" repeatedly\par
\quad - c) To solve complex equations faster\par

\textbf{Q11} Which ancient civilization used geometry to calculate areas of circles and pyramids around 3000 BC?\par
\quad - a) Babylonians\par
\quad - b) Egyptians\par
\quad - c) Greeks\par

\textbf{Q12} Who were the pioneers of solving cubic and quartic equations during the Renaissance?\par
\quad - a) Cardano and Tartaglia\par
\quad - b) Euclid and Pythagoras\par
\quad - c) Newton and Leibniz\par

\textbf{Q13} What discovery by Johannes Kepler revolutionized our understanding of the universe?\par
\quad - a) Laws of planetary motion\par
\quad - b) Cartesian coordinates\par
\quad - c) Group theory\par

\textbf{Q14} Calculus was independently developed by Isaac Newton and …\par
\quad - a) René Descartes\par
\quad - b) Gottfried Wilhelm Leibniz\par
\quad - c) John Napier\par

\textbf{Q15} What is an everyday application of advanced mathematics?\par
\quad - a) Designing furniture layouts\par
\quad - b) Writing poetry\par
\quad - c) Painting portraits\par

\textbf{Q16} When was the equals sign (=) invented?\par
\quad - a) 1657\par
\quad - b) 1557\par
\quad - c) 1757\par

\textbf{Q17} Logarithms were developed by …\par
\quad - a) Johannes Kepler\par
\quad - b) John Napier\par
\quad - c) Galileo Galilei\par

\textbf{Q18} Which mathematician's work laid the foundation for modern analytical geometry?\par
\quad - a) René Descartes\par
\quad - b) Évariste Galois\par
\quad - c) Charles Babbage\par

\textbf{Q19} Which civilization independently developed the concept of zero alongside India?\par
\quad - a) Babylonians\par
\quad - b) Mayan\par
\quad - c) Egyptians\par

\textbf{Q20} What discipline emerged during the Italian Renaissance as a formal mathematical field?\par
\quad - a) Algebra\par
\quad - b) Geometry\par
\quad - c) Calculus\par

\textbf{Q21} Which book is considered a foundational text in geometry, introducing axiomatic reasoning?\par
\quad - a) Liber Abaci\par
\quad - b) Euclid's *Elements*\par
\quad - c) *Disquisitiones Arithmeticae*\par

\textbf{Q22} Who is known as the "father of algebra" for his work *Arithmetica*?\par
\quad - a) Fibonacci\par
\quad - b) Diophantus of Alexandria\par
\quad - c) René Descartes\par

\textbf{Q23} Which book introduced the Hindu‑Arabic numeral system to Europe?\par
\quad - a) Liber Abaci\par
\quad - b) Arithmetica\par
\quad - c) *Principia Mathematica*\par

\textbf{Q24} Isaac Newton's *Philosophiae Naturalis Principia Mathematica* is significant for its contribution to …\par
\quad - a) Number theory\par
\quad - b) Classical mechanics\par
\quad - c) Algebraic geometry\par

\textbf{Q25} Which mathematician revolutionized number theory with *Disquisitiones Arithmeticae*?\par
\quad - a) Carl Friedrich Gauss\par
\quad - b) Andrey Kolmogorov\par
\quad - c) Kurt Gödel\par

\textbf{Q26} René Descartes' *La Géométrie* introduced which mathematical concept?\par
\quad - a) Modular arithmetic\par
\quad - b) Cartesian coordinates\par
\quad - c) Fibonacci sequence\par

\textbf{Q27} The book *Principia Mathematica* by Whitehead and Russell focuses on …\par
\quad - a) Quantum mechanics\par
\quad - b) Symbolic logic\par
\quad - c) Probability theory\par

\textbf{Q28} Which book formalized quantum mechanics using Hilbert spaces?\par
\quad - a) *Mathematische Grundlagen der Quantenmechanik*\par
\quad - b) *Grundbegriffe der Wahrscheinlichkeitsrechnung*\par
\quad - c) *Theory of Games and Economic Behavior*\par

\textbf{Q29} The foundation of game theory was established in …\par
\quad - a) Gödel's *On Formally Undecidable Propositions*\par
\quad - b) *Theory of Games and Economic Behavior*\par
\quad - c) *Disquisitiones Arithmeticae*\par

\textbf{Q30} Andrey Kolmogorov's seminal work in probability theory is titled …\par
\quad - a) *Grundbegriffe der Wahrscheinlichkeitsrechnung*\par
\quad - b) *La Géométrie*\par
\quad - c) *Principia Mathematica*\par

\textbf{Q31} Kurt Gödel's incompleteness theorems were published in …\par
\quad - a) *Disquisitiones Arithmeticae*\par
\quad - b) *On Formally Undecidable Propositions*\par
\quad - c) *Elements*\par

\textbf{Q32} Which book gives a comprehensive overview of mathematical developments across cultures?\par
\quad - a) *A History of Mathematics*\par
\quad - b) *Math Through the Ages*\par
\quad - c) *Journey Through Genius*\par

\textbf{Q33} Who wrote *Journey Through Genius: The Great Theorems of Mathematics*?\par
\quad - a) Simon Singh\par
\quad - b) William Dunham\par
\quad - c) Steven Strogatz\par

\textbf{Q34} *Fermat's Enigma* tells the story of which famous theorem?\par
\quad - a) Pythagorean theorem\par
\quad - b) Fermat's Last Theorem\par
\quad - c) Gödel's incompleteness theorem\par

\textbf{Q35} What is the focus of *Math Through the Ages*?\par
\quad - a) Advanced mathematical theories\par
\quad - b) Beginner‑friendly history of mathematics\par
\quad - c) Quantum mechanics\par

\textbf{Q36} Steven Strogatz's book *Infinite Powers* discusses …\par
\quad - a) Calculus and its impact on science\par
\quad - b) Probability theory\par
\quad - c) Game theory\par

\textbf{Q37} *The Joy of x* by Steven Strogatz is a …\par
\quad - a) Satirical novel\par
\quad - b) Lighthearted exploration of math in everyday life\par
\quad - c) Comprehensive reference for professionals\par

\textbf{Q38} *How Not to Be Wrong* by Jordan Ellenberg emphasizes …\par
\quad - a) Applications of math in daily decision‑making\par
\quad - b) The history of geometry\par
\quad - c) Algebra and its foundations\par

\textbf{Q39} *The Princeton Companion to Mathematics* is best suited for …\par
\quad - a) Beginners in mathematics\par
\quad - b) Advanced students and professionals\par
\quad - c) Historians of mathematics\par

\textbf{Q40} *Flatland: A Romance of Many Dimensions* introduces dimensions through …\par
\quad - a) A mathematical proof\par
\quad - b) A fictional two‑dimensional world\par
\quad - c) A set of axioms\par

\textbf{Q41} *Liber Abaci* introduced which famous problem?\par
\quad - a) The Pythagorean theorem\par
\quad - b) The rabbit problem\par
\quad - c) Fermat's Last Theorem\par

\textbf{Q42} Which book is described as a satirical novel about dimensions?\par
\quad - a) *Flatland*\par
\quad - b) *The Joy of x*\par
\quad - c) *Infinite Powers*\par

\textbf{Q43} What is a key topic in *The Princeton Companion to Mathematics*?\par
\quad - a) A beginner's guide to algebra\par
\quad - b) Modern mathematics across disciplines\par
\quad - c) Famous mathematical theorems\par

\textbf{Q44} *Journey Through Genius* discusses …\par
\quad - a) Famous mathematical breakthroughs\par
\quad - b) Practical applications of math\par
\quad - c) Abstract algebra theories\par

\textbf{Q45} Which book formalized the mathematical framework for quantum mechanics?\par
\quad - a) *La Géométrie*\par
\quad - b) *Mathematische Grundlagen der Quantenmechanik*\par
\quad - c) *Principia Mathematica*\par

\textbf{Q46} Who co‑authored *Theory of Games and Economic Behavior*?\par
\quad - a) John von Neumann and Oskar Morgenstern\par
\quad - b) Bertrand Russell and Alfred North Whitehead\par
\quad - c) Isaac Newton and Carl Friedrich Gauss\par

\textbf{Q47} What inspired Fermat's work mentioned in *Arithmetica*?\par
\quad - a) Fibonacci sequence\par
\quad - b) Syncopated algebraic notation\par
\quad - c) Modular arithmetic\par

\textbf{Q48} Gödel's incompleteness theorems demonstrated the limits of …\par
\quad - a) Probability theory\par
\quad - b) Formal mathematical systems\par
\quad - c) Classical mechanics\par

\textbf{Q49} What does *Math Through the Ages* offer?\par
\quad - a) An advanced guide to calculus\par
\quad - b) A historical overview for beginners\par
\quad - c) A mathematical perspective on physics\par

\textbf{Q50} *Infinite Powers* emphasizes the role of …\par
\quad - a) Algebra in geometry\par
\quad - b) Calculus in understanding the universe\par
\quad - c) Symbolic logic in modern math\par