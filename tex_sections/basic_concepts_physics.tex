\section{BASIC CONCEPTS OF PHYSICS}

\textbf{Q1} Why do astronauts aboard the International Space Station seem to float?\par
\quad - a) They are outside the gravitational pull of Earth.\par
\quad - b) They are in free fall, circling Earth at the same velocity as the station.\par
\quad - c) They are moved by conditions of zero gravity.\par

\textbf{Q2} What is the name of the force that stops you from falling through a chair when you sit down?\par
\quad - a) Frictional Force\par
\quad - b) Normal Force\par
\quad - c) Gravitational Force\par

\textbf{Q3} If a vehicle moves at a steady speed on a circular track, what kind of acceleration is it experiencing?\par
\quad - a) Tangential Acceleration\par
\quad - b) Centripetal Acceleration\par
\quad - c) Linear Acceleration\par

\textbf{Q4} What is the famous equation that connects energy and mass, and what do each of the variables signify in E=mc²?\par
\quad - a) Energy, mass, and the speed of sound.\par
\quad - b) Energy, mass, and the speed of light.\par
\quad - c) Energy, mass, and acceleration.\par

\textbf{Q5} What scientific principle accounts for why rockets can launch themselves in the emptiness of space?\par
\quad - a) Conservation of Energy\par
\quad - b) Newton's Third Law of Motion\par
\quad - c) Bernoulli's Principle\par

\textbf{Q6} How is it possible for a steel ship to float on water, given that steel is denser than water?\par
\quad - a) Its hollow design lowers its overall density.\par
\quad - b) Its configuration pushes aside enough water to counterbalance its weight.\par
\quad - c) Steel becomes lighter when heated.\par

\textbf{Q7} What would occur to an object's weight if it were taken to the Moon, where gravity is roughly one‑sixth of Earth's gravity?\par
\quad - a) The object's weight would lessen.\par
\quad - b) The object's mass would diminish.\par
\quad - c) Both weight and mass would remain the same.\par

\textbf{Q8} If two objects of different weights are dropped from the same height in a vacuum, what will happen?\par
\quad - a) The heavier object will reach the ground first.\par
\quad - b) The lighter object will land first.\par
\quad - c) Both objects will touch the ground simultaneously.\par

\textbf{Q9} What type of wave is sound, and how does it move through the air?\par
\quad - a) Transverse wave; through the vibrations of particles that are perpendicular to the direction of the wave.\par
\quad - b) Longitudinal wave; by means of compressions and rarefactions of particles along the wave's direction.\par
\quad - c) Electromagnetic wave; through the oscillation of electric and magnetic fields.\par

\textbf{Q10} Which thermodynamic law states that energy cannot be created or annihilated, only changed?\par
\quad - a) First Law of Thermodynamics\par
\quad - b) Second Law of Thermodynamics\par
\quad - c) Third Law of Thermodynamics\par

\textbf{Q11} What causes your ears to pop when ascending to high altitudes?\par
\quad - a) The reduction in air pressure leads to the expansion of trapped air in your ears.\par
\quad - b) Your eardrums adapt to the decreased levels of oxygen.\par
\quad - c) The altitude exerts a pull on your eardrums.\par

\textbf{Q12} Why does a spinning ice skater bring their arms closer to their body to spin more quickly?\par
\quad - a) To enhance their moment of inertia.\par
\quad - b) To maintain angular momentum.\par
\quad - c) To lessen air resistance.\par

\textbf{Q13} What principle explains the change in pitch of an ambulance siren as it passes by?\par
\quad - a) Refraction\par
\quad - b) Doppler Effect\par
\quad - c) Polarization\par

\textbf{Q14} How is the kinetic energy of an object affected if its velocity is doubled?\par
\quad - a) It becomes twice as much.\par
\quad - b) It becomes four times as much.\par
\quad - c) It remains unchanged.\par

\textbf{Q15} Why does black clothing make you feel warmer than white when it's sunny?\par
\quad - a) Black absorbs more heat by reflecting less sunlight.\par
\quad - b) Black retains air molecules within the fabric.\par
\quad - c) Black radiates heat more efficiently.\par

\textbf{Q16} What leads to the formation of a rainbow after rain?\par
\quad - a) Diffraction of sunlight passing through raindrops.\par
\quad - b) Reflection and refraction of sunlight inside the raindrops.\par
\quad - c) Polarization of sunlight by water.\par

\textbf{Q17} Why do you feel lighter when in an elevator that is accelerating downward?\par
\quad - a) The elevator negates gravity's effect.\par
\quad - b) The normal force acting on you decreases.\par
\quad - c) Your weight briefly becomes zero.\par

\textbf{Q18} What force keeps satellites orbiting the Earth rather than drifting off into space?\par
\quad - a) Friction\par
\quad - b) Gravity\par
\quad - c) Centrifugal Force\par

\textbf{Q19} Why does a metal spoon feel hotter than a wooden spoon when touching a hot bowl of soup?\par
\quad - a) Metal is a superior heat conductor.\par
\quad - b) Metal molecules vibrate more rapidly than those in wood.\par
\quad - c) Metal absorbs heat faster than wood does.\par

\textbf{Q20} What causes objects to appear smaller as they move further away?\par
\quad - a) Perspective diminishes their angular size.\par
\quad - b) The speed of light alters their image.\par
\quad - c) Atmospheric particles absorb light.\par

\textbf{Q21} How does a lens in a magnifying glass manipulate light to enlarge objects?\par
\quad - a) It absorbs shorter light wavelengths.\par
\quad - b) It converges parallel light rays to a single point.\par
\quad - c) It reflects light back towards the object.\par

\textbf{Q22} If you move forward in a train at 5 m/s, while the train itself travels at 30 m/s, what is your total speed relative to the ground?\par
\quad - a) 25 m/s\par
\quad - b) 35 m/s\par
\quad - c) 30 m/s\par

\textbf{Q23} What occurrence prevents the Sun's rays from uniformly warming the Earth, leading to seasonal changes?\par
\quad - a) The elliptical trajectory of Earth around the Sun.\par
\quad - b) The tilt of Earth's axis.\par
\quad - c) The variable energy output from the Sun.\par

\textbf{Q24} Why do raindrops take on a spherical shape as they descend through the air?\par
\quad - a) Atmospheric pressure shapes them into a sphere.\par
\quad - b) Surface tension reduces their surface area.\par
\quad - c) Gravity pulls the water into spherical forms.\par

\textbf{Q25} Why does light bend when transitioning from water to air?\par
\quad - a) The water's density slows light down, causing it to speed up again in air.\par
\quad - b) Water refracts light more significantly than air does.\par
\quad - c) Light tends to travel in straight lines only when in air.\par

\textbf{Q26} What is the term for the minimum amount of energy required for a particle to escape the gravitational pull of a celestial body?\par
\quad - a) Escape Velocity\par
\quad - b) Orbital Velocity\par
\quad - c) Terminal Velocity\par

\textbf{Q27} How does the principle of buoyancy explain why objects float or sink in fluids?\par
\quad - a) Objects float if they are denser than the fluid.\par
\quad - b) Objects sink if they are less dense than the fluid.\par
\quad - c) Objects float if they displace a volume of fluid equal to their weight.\par

\textbf{Q28} What phenomenon occurs when light passes from one medium to another and changes direction?\par
\quad - a) Reflection\par
\quad - b) Refraction\par
\quad - c) Diffraction\par

\textbf{Q29} Which type of electromagnetic wave has the highest frequency and shortest wavelength?\par
\quad - a) Radio Waves\par
\quad - b) Gamma Rays\par
\quad - c) Infrared Waves\par

\textbf{Q30} What is the term for the energy transferred between objects due to a temperature difference?\par
\quad - a) Conduction\par
\quad - b) Convection\par
\quad - c) Radiation\par

\textbf{Q31} Why does a bicycle remain upright when moving at a steady speed?\par
\quad - a) Due to gyroscopic effect\par
\quad - b) Due to frictional forces\par
\quad - c) Due to angular momentum\par

\textbf{Q32} What is the principle behind the operation of a refrigerator?\par
\quad - a) Second Law of Thermodynamics\par
\quad - b) First Law of Thermodynamics\par
\quad - c) Heat Transfer\par

\textbf{Q33} How does a prism separate white light into its component colors?\par
\quad - a) Through reflection\par
\quad - b) Through refraction and dispersion\par
\quad - c) Through diffraction\par

\textbf{Q34} What is the term for the force that opposes motion between two surfaces in contact?\par
\quad - a) Normal Force\par
\quad - b) Frictional Force\par
\quad - c) Gravitational Force\par

\textbf{Q35} Why does a hot air balloon rise into the air?\par
\quad - a) Due to buoyancy\par
\quad - b) Due to air pressure\par
\quad - c) Due to wind currents\par

\textbf{Q36} What is the principle behind the operation of a microwave oven?\par
\quad - a) Electromagnetic induction\par
\quad - b) Dielectric heating\par
\quad - c) Convection heating\par

\textbf{Q37} How does a concave mirror focus light?\par
\quad - a) By diverging light rays\par
\quad - b) By converging light rays\par
\quad - c) By reflecting light back to its source\par

\textbf{Q38} What is the term for the minimum speed required for an object to orbit the Earth?\par
\quad - a) Escape Velocity\par
\quad - b) Orbital Velocity\par
\quad - c) Terminal Velocity\par

\textbf{Q39} Why does a compass needle align itself with the Earth's magnetic field?\par
\quad - a) Due to gravitational forces\par
\quad - b) Due to magnetic forces\par
\quad - c) Due to electromagnetic induction\par

\textbf{Q40} What is the principle behind the operation of a laser?\par
\quad - a) Stimulated emission\par
\quad - b) Spontaneous emission\par
\quad - c) Absorption of light\par

\textbf{Q41} How does a telescope magnify distant objects?\par
\quad - a) By reflecting light\par
\quad - b) By refracting light\par
\quad - c) By converging light rays\par

\textbf{Q42} What is the term for the energy stored in an object due to its position or configuration?\par
\quad - a) Kinetic Energy\par
\quad - b) Potential Energy\par
\quad - c) Thermal Energy\par

\textbf{Q43} Why does a parachute slow down the fall of an object?\par
\quad - a) Due to air resistance\par
\quad - b) Due to gravity\par
\quad - c) Due to buoyancy\par

\textbf{Q44} What is the principle behind the operation of a wind turbine?\par
\quad - a) Conservation of Energy\par
\quad - b) Conversion of kinetic energy to electrical energy\par
\quad - c) Bernoulli's Principle\par

\textbf{Q45} How does a magnet attract certain metals?\par
\quad - a) Due to gravitational forces\par
\quad - b) Due to magnetic forces\par
\quad - c) Due to electromagnetic induction\par

\textbf{Q46} What is the term for the process by which heat is transferred through direct contact?\par
\quad - a) Conduction\par
\quad - b) Convection\par
\quad - c) Radiation\par

\textbf{Q47} Why does a mirror reflect light?\par
\quad - a) Due to refraction\par
\quad - b) Due to reflection\par
\quad - c) Due to diffraction\par

\textbf{Q48} What is the principle behind the operation of a solar panel?\par
\quad - a) Photovoltaic effect\par
\quad - b) Thermoelectric effect\par
\quad - c) Electromagnetic induction\par

\textbf{Q49} How does a bicycle wheel maintain its stability when moving?\par
\quad - a) Due to gyroscopic effect\par
\quad - b) Due to frictional forces\par
\quad - c) Due to angular momentum\par

\textbf{Q50} What is the term for the energy associated with the motion of an object?\par
\quad - a) Potential Energy\par
\quad - b) Kinetic Energy\par
\quad - c) Thermal Energy\par