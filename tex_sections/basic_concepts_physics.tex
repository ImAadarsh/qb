
\begin{enhancedmcq}{Why do astronauts aboard the International Space Station seem to float?}
\item They are outside the gravitational pull of Earth.
\item They are in free fall, circling Earth at the same velocity as the station.
\item They are moved by conditions of zero gravity.

\end{enhancedmcq}
\begin{enhancedmcq}{What is the name of the force that stops you from falling through a chair when you sit down?}
\item Frictional Force
\item Normal Force
\item Gravitational Force

\end{enhancedmcq}
\begin{enhancedmcq}{If a vehicle moves at a steady speed on a circular track, what kind of acceleration is it experiencing?}
\item Tangential Acceleration
\item Centripetal Acceleration
\item Linear Acceleration

\end{enhancedmcq}
\begin{enhancedmcq}{What is the famous equation that connects energy and mass, and what do each of the variables signify in E=mc²?}
\item Energy, mass, and the speed of sound.
\item Energy, mass, and the speed of light.
\item Energy, mass, and acceleration.

\end{enhancedmcq}
\begin{enhancedmcq}{What scientific principle accounts for why rockets can launch themselves in the emptiness of space?}
\item Conservation of Energy
\item Newton's Third Law of Motion
\item Bernoulli's Principle

\end{enhancedmcq}
\begin{enhancedmcq}{How is it possible for a steel ship to float on water, given that steel is denser than water?}
\item Its hollow design lowers its overall density.
\item Its configuration pushes aside enough water to counterbalance its weight.
\item Steel becomes lighter when heated.

\end{enhancedmcq}
\begin{enhancedmcq}{What would occur to an object's weight if it were taken to the Moon, where gravity is roughly one‑sixth of Earth's gravity?}
\item The object's weight would lessen.
\item The object's mass would diminish.
\item Both weight and mass would remain the same.

\end{enhancedmcq}
\begin{enhancedmcq}{If two objects of different weights are dropped from the same height in a vacuum, what will happen?}
\item The heavier object will reach the ground first.
\item The lighter object will land first.
\item Both objects will touch the ground simultaneously.

\end{enhancedmcq}
\begin{enhancedmcq}{What type of wave is sound, and how does it move through the air?}
\item Transverse wave; through the vibrations of particles that are perpendicular to the direction of the wave.
\item Longitudinal wave; by means of compressions and rarefactions of particles along the wave's direction.
\item Electromagnetic wave; through the oscillation of electric and magnetic fields.

\end{enhancedmcq}
\begin{enhancedmcq}{Which thermodynamic law states that energy cannot be created or annihilated, only changed?}
\item First Law of Thermodynamics
\item Second Law of Thermodynamics
\item Third Law of Thermodynamics

\end{enhancedmcq}
\begin{enhancedmcq}{What causes your ears to pop when ascending to high altitudes?}
\item The reduction in air pressure leads to the expansion of trapped air in your ears.
\item Your eardrums adapt to the decreased levels of oxygen.
\item The altitude exerts a pull on your eardrums.

\end{enhancedmcq}
\begin{enhancedmcq}{Why does a spinning ice skater bring their arms closer to their body to spin more quickly?}
\item To enhance their moment of inertia.
\item To maintain angular momentum.
\item To lessen air resistance.

\end{enhancedmcq}
\begin{enhancedmcq}{What principle explains the change in pitch of an ambulance siren as it passes by?}
\item Refraction
\item Doppler Effect
\item Polarization

\end{enhancedmcq}
\begin{enhancedmcq}{How is the kinetic energy of an object affected if its velocity is doubled?}
\item It becomes twice as much.
\item It becomes four times as much.
\item It remains unchanged.

\end{enhancedmcq}
\begin{enhancedmcq}{Why does black clothing make you feel warmer than white when it's sunny?}
\item Black absorbs more heat by reflecting less sunlight.
\item Black retains air molecules within the fabric.
\item Black radiates heat more efficiently.

\end{enhancedmcq}
\begin{enhancedmcq}{What leads to the formation of a rainbow after rain?}
\item Diffraction of sunlight passing through raindrops.
\item Reflection and refraction of sunlight inside the raindrops.
\item Polarization of sunlight by water.

\end{enhancedmcq}
\begin{enhancedmcq}{Why do you feel lighter when in an elevator that is accelerating downward?}
\item The elevator negates gravity's effect.
\item The normal force acting on you decreases.
\item Your weight briefly becomes zero.

\end{enhancedmcq}
\begin{enhancedmcq}{What force keeps satellites orbiting the Earth rather than drifting off into space?}
\item Friction
\item Gravity
\item Centrifugal Force

\end{enhancedmcq}
\begin{enhancedmcq}{Why does a metal spoon feel hotter than a wooden spoon when touching a hot bowl of soup?}
\item Metal is a superior heat conductor.
\item Metal molecules vibrate more rapidly than those in wood.
\item Metal absorbs heat faster than wood does.

\end{enhancedmcq}
\begin{enhancedmcq}{What causes objects to appear smaller as they move further away?}
\item Perspective diminishes their angular size.
\item The speed of light alters their image.
\item Atmospheric particles absorb light.

\end{enhancedmcq}
\begin{enhancedmcq}{How does a lens in a magnifying glass manipulate light to enlarge objects?}
\item It absorbs shorter light wavelengths.
\item It converges parallel light rays to a single point.
\item It reflects light back towards the object.

\end{enhancedmcq}
\begin{enhancedmcq}{If you move forward in a train at 5 m/s, while the train itself travels at 30 m/s, what is your total speed relative to the ground?}
\item 25 m/s
\item 35 m/s
\item 30 m/s

\end{enhancedmcq}
\begin{enhancedmcq}{What occurrence prevents the Sun's rays from uniformly warming the Earth, leading to seasonal changes?}
\item The elliptical trajectory of Earth around the Sun.
\item The tilt of Earth's axis.
\item The variable energy output from the Sun.

\end{enhancedmcq}
\begin{enhancedmcq}{Why do raindrops take on a spherical shape as they descend through the air?}
\item Atmospheric pressure shapes them into a sphere.
\item Surface tension reduces their surface area.
\item Gravity pulls the water into spherical forms.

\end{enhancedmcq}
\begin{enhancedmcq}{Why does light bend when transitioning from water to air?}
\item The water's density slows light down, causing it to speed up again in air.
\item Water refracts light more significantly than air does.
\item Light tends to travel in straight lines only when in air.

\end{enhancedmcq}
\begin{enhancedmcq}{What is the term for the minimum amount of energy required for a particle to escape the gravitational pull of a celestial body?}
\item Escape Velocity
\item Orbital Velocity
\item Terminal Velocity

\end{enhancedmcq}
\begin{enhancedmcq}{How does the principle of buoyancy explain why objects float or sink in fluids?}
\item Objects float if they are denser than the fluid.
\item Objects sink if they are less dense than the fluid.
\item Objects float if they displace a volume of fluid equal to their weight.

\end{enhancedmcq}
\begin{enhancedmcq}{What phenomenon occurs when light passes from one medium to another and changes direction?}
\item Reflection
\item Refraction
\item Diffraction

\end{enhancedmcq}
\begin{enhancedmcq}{Which type of electromagnetic wave has the highest frequency and shortest wavelength?}
\item Radio Waves
\item Gamma Rays
\item Infrared Waves

\end{enhancedmcq}
\begin{enhancedmcq}{What is the term for the energy transferred between objects due to a temperature difference?}
\item Conduction
\item Convection
\item Radiation

\end{enhancedmcq}
\begin{enhancedmcq}{Why does a bicycle remain upright when moving at a steady speed?}
\item Due to gyroscopic effect
\item Due to frictional forces
\item Due to angular momentum

\end{enhancedmcq}
\begin{enhancedmcq}{What is the principle behind the operation of a refrigerator?}
\item Second Law of Thermodynamics
\item First Law of Thermodynamics
\item Heat Transfer

\end{enhancedmcq}
\begin{enhancedmcq}{How does a prism separate white light into its component colors?}
\item Through reflection
\item Through refraction and dispersion
\item Through diffraction

\end{enhancedmcq}
\begin{enhancedmcq}{What is the term for the force that opposes motion between two surfaces in contact?}
\item Normal Force
\item Frictional Force
\item Gravitational Force

\end{enhancedmcq}
\begin{enhancedmcq}{Why does a hot air balloon rise into the air?}
\item Due to buoyancy
\item Due to air pressure
\item Due to wind currents

\end{enhancedmcq}
\begin{enhancedmcq}{What is the principle behind the operation of a microwave oven?}
\item Electromagnetic induction
\item Dielectric heating
\item Convection heating

\end{enhancedmcq}
\begin{enhancedmcq}{How does a concave mirror focus light?}
\item By diverging light rays
\item By converging light rays
\item By reflecting light back to its source

\end{enhancedmcq}
\begin{enhancedmcq}{What is the term for the minimum speed required for an object to orbit the Earth?}
\item Escape Velocity
\item Orbital Velocity
\item Terminal Velocity

\end{enhancedmcq}
\begin{enhancedmcq}{Why does a compass needle align itself with the Earth's magnetic field?}
\item Due to gravitational forces
\item Due to magnetic forces
\item Due to electromagnetic induction

\end{enhancedmcq}
\begin{enhancedmcq}{What is the principle behind the operation of a laser?}
\item Stimulated emission
\item Spontaneous emission
\item Absorption of light

\end{enhancedmcq}
\begin{enhancedmcq}{How does a telescope magnify distant objects?}
\item By reflecting light
\item By refracting light
\item By converging light rays

\end{enhancedmcq}
\begin{enhancedmcq}{What is the term for the energy stored in an object due to its position or configuration?}
\item Kinetic Energy
\item Potential Energy
\item Thermal Energy

\end{enhancedmcq}
\begin{enhancedmcq}{Why does a parachute slow down the fall of an object?}
\item Due to air resistance
\item Due to gravity
\item Due to buoyancy

\end{enhancedmcq}
\begin{enhancedmcq}{What is the principle behind the operation of a wind turbine?}
\item Conservation of Energy
\item Conversion of kinetic energy to electrical energy
\item Bernoulli's Principle

\end{enhancedmcq}
\begin{enhancedmcq}{How does a magnet attract certain metals?}
\item Due to gravitational forces
\item Due to magnetic forces
\item Due to electromagnetic induction

\end{enhancedmcq}
\begin{enhancedmcq}{What is the term for the process by which heat is transferred through direct contact?}
\item Conduction
\item Convection
\item Radiation

\end{enhancedmcq}
\begin{enhancedmcq}{Why does a mirror reflect light?}
\item Due to refraction
\item Due to reflection
\item Due to diffraction

\end{enhancedmcq}
\begin{enhancedmcq}{What is the principle behind the operation of a solar panel?}
\item Photovoltaic effect
\item Thermoelectric effect
\item Electromagnetic induction

\end{enhancedmcq}
\begin{enhancedmcq}{How does a bicycle wheel maintain its stability when moving?}
\item Due to gyroscopic effect
\item Due to frictional forces
\item Due to angular momentum

\end{enhancedmcq}
\begin{enhancedmcq}{What is the term for the energy associated with the motion of an object?}
\item Potential Energy
\item Kinetic Energy
\item Thermal Energy
\end{enhancedmcq}