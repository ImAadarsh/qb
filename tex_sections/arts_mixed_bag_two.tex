
\begin{enhancedmcq}{Guru Ghasidas National Park is named after the founder of the Satnami sect. What were the main teachings of Guru Ghasidas?}
\item Monotheism and social equality
\item Polytheism and caste hierarchy
\item Non‑violence and asceticism

\end{enhancedmcq}
\begin{enhancedmcq}{Swami Vivekananda set sail for India in 1896. Which conference is he best known for attending before this voyage?}
\item World Parliament of Religions
\item Round Table Conference
\item Bandung Conference

\end{enhancedmcq}
\begin{enhancedmcq}{Ram Prasad Bismil wrote "Sarfaroshi ki Tamanna." What was the main message of this patriotic song?}
\item Peaceful resistance
\item Revolutionary zeal
\item Social reform

\end{enhancedmcq}
\begin{enhancedmcq}{Mehrangarh Fort is located in Jodhpur. What is the main architectural style of this fort?}
\item Mughal
\item Rajput
\item Dravidian

\end{enhancedmcq}
\begin{enhancedmcq}{Jama Masjid in Delhi houses relics of the Prophet. Which Mughal emperor commissioned the construction of this mosque?}
\item Akbar
\item Shah Jahan
\item Aurangzeb

\end{enhancedmcq}
\begin{enhancedmcq}{Sain Hazrat Mian Mir laid the foundation stone of the Golden Temple. What is his main legacy?}
\item He was a Sufi saint known for interfaith harmony
\item He was a Mughal courtier
\item He was a Sikh warrior

\end{enhancedmcq}
\begin{enhancedmcq}{Dachigam National Park is known for the Marsar Lake. Which animal is this park primarily known for protecting?}
\item Snow Leopard
\item Hangul (Kashmiri Stag)
\item Red Panda

\end{enhancedmcq}
\begin{enhancedmcq}{Darjeeling Tea was the first Indian product to receive a GI tag. What does a GI tag primarily protect?}
\item Manufacturing process
\item Geographical origin and quality
\item Brand name

\end{enhancedmcq}
\begin{enhancedmcq}{The Bishnoi sect was founded by Guru Jambeshwar. What is the core principle of this sect?}
\item Protecting the environment and wildlife
\item Promoting trade and commerce
\item Advocating for women's rights

\end{enhancedmcq}
\begin{enhancedmcq}{Gopal Swarup Pathak was the first Indian Vice‑President not to become President. What was his primary profession before entering politics?}
\item Lawyer
\item Doctor
\item Engineer

\end{enhancedmcq}
\begin{enhancedmcq}{Michael Madhusudan Dutta wrote "Hector Badh Kavya." What is the main theme of this play?}
\item Romantic love
\item Heroic sacrifice
\item Political intrigue

\end{enhancedmcq}
\begin{enhancedmcq}{The Chocolate Bomb was invented in Kolkata. What is the main ingredient of this sweet?}
\item Chocolate
\item Coconut
\item Khoya

\end{enhancedmcq}
\begin{enhancedmcq}{Khushwant Singh wrote "The Freethinker's Prayer Book." What is the main focus of his writings?}
\item Spirituality and religion
\item Secularism and humanism
\item History and politics

\end{enhancedmcq}
\begin{enhancedmcq}{Kodaikanal is associated with several landmarks. What is Coaker's Walk known for?}
\item Hiking trail with scenic views
\item Ancient temple
\item Wildlife sanctuary

\end{enhancedmcq}
\begin{enhancedmcq}{Union Bank of India has a tagline. What does the tagline actually mean?}
\item They help you earn more interest
\item They are ethical and dependable
\item They will help you get high returns on stocks

\end{enhancedmcq}
\begin{enhancedmcq}{Guru Hargobind established Akal Takht. What is the significance of Akal Takht?}
\item It is a place for religious ceremonies
\item It is a symbol of Sikh sovereignty
\item It is a center for learning

\end{enhancedmcq}
\begin{enhancedmcq}{The Indian Institute of Science‑Bangalore is ranked in the Global Employability List. What does this ranking signify?}
\item High research output
\item Strong industry connections
\item Excellent infrastructure

\end{enhancedmcq}
\begin{enhancedmcq}{Catlanchimauli is the highest peak in Goa. What type of terrain characterizes this peak?}
\item Rocky and barren
\item Forested and green
\item Snow‑covered

\end{enhancedmcq}
\begin{enhancedmcq}{Gabbar Singh appeared in a Britannia advertisement. Why was this character chosen for the advertisement?}
\item To promote bravery
\item To create a memorable and humorous campaign
\item To appeal to children

\end{enhancedmcq}
\begin{enhancedmcq}{ISRO's Cartosat series is used for mapping. What is the main application of these satellites?}
\item Weather forecasting
\item Earth observation and mapping
\item Telecommunications

\end{enhancedmcq}
\begin{enhancedmcq}{The MKSS campaigned for transparency. Which right was the focus of their campaign?}
\item Right to Education
\item Right to Information
\item Right to Employment

\end{enhancedmcq}
\begin{enhancedmcq}{Leh is a predominantly Buddhist district. Which form of Buddhism is primarily practiced here?}
\item Theravada
\item Mahayana
\item Vajrayana

\end{enhancedmcq}
\begin{enhancedmcq}{Franck Ribery won the UEFA Best Player in Europe Award. Which club did he play for at the time?}
\item Real Madrid
\item Barcelona
\item Bayern Munich

\end{enhancedmcq}
\begin{enhancedmcq}{The first SMS was sent in 1992. What was the purpose of the first ever SMS?}
\item It was just a way to test the new messaging system
\item It was to show what the new technology was capable of
\item It was to wish a merry Christmas

\end{enhancedmcq}
\begin{enhancedmcq}{The movie "Guide" was based on a novel. Which theme was prevalent in the storyline?}
\item A guide falling in love with his tutee's wife
\item A guide solving a murder
\item A guide on a holy road

\end{enhancedmcq}
\begin{enhancedmcq}{Avani Chaturvedi was the first solo pilot in the Indian Air Force. What was her role?}
\item Transport of ammunition to a different base
\item Fighter pilot
\item Surveillance pilot

\end{enhancedmcq}
\begin{enhancedmcq}{What is the meaning of the word "Shakti" in context with India and France?}
\item A military pact against enemies
\item Power through solidarity
\item The army strength

\end{enhancedmcq}
\begin{enhancedmcq}{Who was the first to conceptualize the Indian space program?}
\item C.V. Raman
\item Vikram Sarabhai
\item Homi Bhabha

\end{enhancedmcq}
\begin{enhancedmcq}{Why was the INS Arihant a turning point in the Indian Navy?}
\item It was the biggest ship made by the Navy till then
\item It was the first submarine that was run by nuclear energy
\item It had stealth technology and could not be detected by enemy radar

\end{enhancedmcq}
\begin{enhancedmcq}{Why is Kerala known as the "spice garden of India"?}
\item Its spices are mostly exported to various places in India
\item It is the only place in India where spices are grown
\item The rich soil is very conducive for growing spices

\end{enhancedmcq}
\begin{enhancedmcq}{Bangalore is known as the "Silicon Valley of India" because...}
\item It is the biggest manufacturer of silicon microchips
\item It is the hub for all software technology companies
\item It is closest to the equator

\end{enhancedmcq}
\begin{enhancedmcq}{"The land of five rivers" is also the place that the holiest site for Sikhs is. Which state are we speaking about?}
\item Haryana
\item Punjab
\item Himachal Pradesh

\end{enhancedmcq}
\begin{enhancedmcq}{What happens to the lives of the main characters in "A Suitable Boy"?}
\item They all have tragic endings
\item They all end up wealthy
\item Their lives are intertwined and they all find some sort of fulfilment

\end{enhancedmcq}
\begin{enhancedmcq}{What instrument was Ustad Bismillah Khan most famous for playing?}
\item The Sitar
\item The Tabla
\item The Shehnai

\end{enhancedmcq}
\begin{enhancedmcq}{Which is the highest mountain peak completely within the Indian border?}
\item Mt. K2
\item Kanchenjunga
\item Nanda Devi

\end{enhancedmcq}
\begin{enhancedmcq}{What makes Jim Corbett National Park so suitable for Bengal Tigers?}
\item Its location in the Himalayas ensures that the weather suits their fur and appetite
\item It has a diverse population of various animals who serve as prey for the tigers
\item It is the only dense forest in India

\end{enhancedmcq}
\begin{enhancedmcq}{What is the main industry for which the "City of Joy" is known?}
\item Leather tanneries
\item Software companies
\item Film

\end{enhancedmcq}
\begin{enhancedmcq}{How does the story of "Shakuntala" end?}
\item She dies because of heartbreak
\item She is reunited with her husband
\item She starts a revolution against her husband

\end{enhancedmcq}
\begin{enhancedmcq}{Andhra Pradesh has good land and abundant water because...}
\item There are many lakes
\item There are good monsoons and a river delta
\item The black cotton soil retains water easily

\end{enhancedmcq}
\begin{enhancedmcq}{How is ISRO planning to use the data that they get from the satellites?}
\item Use it for space exploration
\item Use the data for helping the country with agriculture and disaster relief
\item Use the data to spy on other countries

\end{enhancedmcq}
\begin{enhancedmcq}{Why is the city of Jaipur so pink?}
\item It's the favourite colour of the royalty
\item All the buildings and structures were painted pink by order of the King
\item Pink is the cheapest paint in the region

\end{enhancedmcq}
\begin{enhancedmcq}{What exactly were the "Wings of Fire" that Dr. A.P.J. Abdul Kalam was talking about?}
\item The wings of a commercial jetliner that he used to fly in often
\item The project that he contributed to with DRDO
\item The nuclear missiles that were developed

\end{enhancedmcq}
\begin{enhancedmcq}{How does the Damodar River cause sorrow to the state?}
\item It causes frequent landslides
\item It causes frequent floods
\item It causes the water table to drop in the state

\end{enhancedmcq}
\begin{enhancedmcq}{Is the Indian Parliament bicameral or unicameral?}
\item Bicameral
\item Unicameral
\item There is no parliament in India

\end{enhancedmcq}
\begin{enhancedmcq}{Why is Himachal Pradesh so suitable for orchards?}
\item Its proximity to the ocean keeps the weather humid
\item Its high altitude and favourable climate is suitable for growing fruits
\item There is a variety of bees that keep the flowering plants pollinated

\end{enhancedmcq}
\begin{enhancedmcq}{Is there a specific thing that Arundhati Roy wanted to highlight in "The God of Small Things"?}
\item It discusses the issue of forbidden love
\item It talks about child abuse
\item It shows the impact of war on small children

\end{enhancedmcq}
\begin{enhancedmcq}{Which of the leather cities of India has a river flowing through it?}
\item Kanpur
\item Agra
\item Chennai

\end{enhancedmcq}
\begin{enhancedmcq}{Which colour of the Indian national flag is on top and is the strongest symbol of the flag?}
\item White
\item Green
\item Saffron

\end{enhancedmcq}
\begin{enhancedmcq}{What does "Arunachal Pradesh" mean, when literally translated?}
\item The land of the rising sun
\item The land that has a lot of mountains
\item The land where a lot of rain falls

\end{enhancedmcq}
\begin{enhancedmcq}{The story "Train to Pakistan" is about...}
\item A person taking a train to a cricket match
\item The impact of partition on the lives of ordinary people
\item Someone who has to travel for work by train very often
\end{enhancedmcq}