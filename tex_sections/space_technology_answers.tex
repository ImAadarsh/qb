\section{ANSWER KEY: Space Technology}

\textbf{Q1} b) It was the first artificial satellite to orbit Earth\par
\textbf{Q2} a) Yuri Gagarin – Vostok 1\par
\textbf{Q3} b) United States\par
\textbf{Q4} b) To land humans on the Moon and return them safely to Earth\par
\textbf{Q5} b) 1958 – To conduct space and aeronautics research\par
\textbf{Q6} b) Apollo 11\par
\textbf{Q7} b) ISRO (Indian Space Research Organisation)\par
\textbf{Q8} a) To colonize Mars and make humanity multi‑planetary\par
\textbf{Q9} b) Dragon C2+ (SpaceX Dragon Capsule)\par
\textbf{Q10} b) Valentina Tereshkova\par
\textbf{Q11} b) Mariner 9\par
\textbf{Q12} b) Perseverance\par
\textbf{Q13} b) Soviet Union\par
\textbf{Q14} b) To capture high‑resolution images of distant galaxies and stars\par
\textbf{Q15} a) Mangalyaan\par
\textbf{Q16} b) John Glenn\par
\textbf{Q17} b) Voyager 1\par
\textbf{Q18} a) Developing reusable rockets for commercial space tourism\par
\textbf{Q19} b) To study the outer Solar System and beyond\par
\textbf{Q20} b) ESA (European Space Agency)\par
\textbf{Q21} a) Arthur C. Clarke\par
\textbf{Q22} b) Telstar 1\par
\textbf{Q23} a) NASA's Curiosity Rover\par
\textbf{Q24} a) To provide global satellite internet coverage\par
\textbf{Q25} a) Scott Kelly\par
\textbf{Q26} a) Voyager 1\par
\textbf{Q27} c) Tiangong Space Station (TSS)\par
\textbf{Q28} a) Richard Branson\par
\textbf{Q29} a) To conduct long‑duration spaceflight experiments\par
\textbf{Q30} b) ESA (European Space Agency)\par
\textbf{Q31} a) Alexei Leonov\par
\textbf{Q32} a) Artemis\par
\textbf{Q33} a) SpaceX\par
\textbf{Q34} b) To study the geology of Mars\par
\textbf{Q35} a) Robert Goddard\par
\textbf{Q36} a) New Horizons\par
\textbf{Q37} b) ADITYA‑L1\par
\textbf{Q38} a) Gennady Padalka\par
\textbf{Q39} c) NASA and ESA collaboration\par
\textbf{Q40} a) To explore Jupiter's moon Europa for signs of life\par