
\begin{enhancedmcq}{Which Swedish botanist developed the binomial nomenclature system used to classify organisms?}
\item Carl Linnaeus
\item Gregor Mendel
\item Charles Darwin

\end{enhancedmcq}
\begin{enhancedmcq}{What unique adaptation allows tardigrades (water bears) to survive in extreme environments, including the vacuum of space?}
\item Cryptobiosis
\item Bioluminescence
\item Symbiosis

\end{enhancedmcq}
\begin{enhancedmcq}{Which scientist conducted the famous pea plant experiments that established the basic principles of heredity?}
\item Gregor Mendel
\item Thomas Hunt Morgan
\item Barbara McClintock

\end{enhancedmcq}
\begin{enhancedmcq}{What is the only known organelle that contains its own separate DNA independent from the cell nucleus?}
\item Mitochondria
\item Golgi apparatus
\item Endoplasmic reticulum

\end{enhancedmcq}
\begin{enhancedmcq}{Which female scientist's X‑ray crystallography work was crucial to discovering the DNA double helix structure?}
\item Rosalind Franklin
\item Marie Curie
\item Barbara McClintock

\end{enhancedmcq}
\begin{enhancedmcq}{What is the term for animals that can regenerate entire body parts or organs?}
\item Autotomous
\item Regenerative
\item Pluripotent

\end{enhancedmcq}
\begin{enhancedmcq}{Which neuroscientist was awarded the Nobel Prize for discovering split‑brain phenomenon?}
\item Roger Sperry
\item Santiago Ramón y Cajal
\item Eric Kandel

\end{enhancedmcq}
\begin{enhancedmcq}{What animal has the largest brain‑to‑body‑weight ratio of any vertebrate?}
\item Hummingbird
\item Dolphin
\item Human

\end{enhancedmcq}
\begin{enhancedmcq}{What is the name of the phenomenon where an isolated population develops unique adaptations?}
\item Adaptive radiation
\item Genetic drift
\item Allopatric speciation

\end{enhancedmcq}
\begin{enhancedmcq}{Which scientist first described the cell nucleus in 1831?}
\item Robert Brown
\item Anton van Leeuwenhoek
\item Robert Hooke

\end{enhancedmcq}
\begin{enhancedmcq}{What is the enzyme that repairs breaks in DNA during replication?}
\item DNA ligase
\item DNA polymerase
\item Helicase

\end{enhancedmcq}
\begin{enhancedmcq}{Which scientist's observation of finches on the Galapagos Islands greatly influenced his theory of evolution?}
\item Charles Darwin
\item Alfred Russel Wallace
\item Jean‑Baptiste Lamarck

\end{enhancedmcq}
\begin{enhancedmcq}{What unusual biological phenomenon allows some reptiles to reproduce without fertilization?}
\item Parthenogenesis
\item Hermaphroditism
\item Polyembryony

\end{enhancedmcq}
\begin{enhancedmcq}{Which microbiologist developed the first effective vaccine against rabies?}
\item Louis Pasteur
\item Robert Koch
\item Edward Jenner

\end{enhancedmcq}
\begin{enhancedmcq}{What is the process called when a species evolves to resemble another unrelated species?}
\item Mimicry
\item Convergent evolution
\item Divergent evolution

\end{enhancedmcq}
\begin{enhancedmcq}{Which scientist is credited with discovering the concept of natural selection simultaneously with Darwin?}
\item Alfred Russel Wallace
\item Thomas Huxley
\item Herbert Spencer

\end{enhancedmcq}
\begin{enhancedmcq}{What type of RNA carries amino acids to ribosomes during protein synthesis?}
\item Transfer RNA (tRNA)
\item Messenger RNA (mRNA)
\item Ribosomal RNA (rRNA)

\end{enhancedmcq}
\begin{enhancedmcq}{Which animal's blood contains copper instead of iron, making it blue instead of red?}
\item Bluefish
\item Octopus
\item Greyseal

\end{enhancedmcq}
\begin{enhancedmcq}{Which biochemist discovered how to determine the complete amino acid sequence of insulin?}
\item Frederick Sanger
\item James Watson
\item Linus Pauling

\end{enhancedmcq}
\begin{enhancedmcq}{What is the term for plants that grow on other plants without being parasitic?}
\item Epiphytes
\item Saprophytes
\item Xerophytes

\end{enhancedmcq}
\begin{enhancedmcq}{Which scientist first described the process of photosynthesis in plants?}
\item Jan Ingenhousz
\item Joseph Priestley
\item Jean Senebier

\end{enhancedmcq}
\begin{enhancedmcq}{What type of cell division results in genetically identical daughter cells?}
\item Mitosis
\item Meiosis
\item Binary fission

\end{enhancedmcq}
\begin{enhancedmcq}{Which woman received the Nobel Prize for her discovery of mobile genetic elements (jumping genes)?}
\item Barbara McClintock
\item Lynn Margulis
\item Nettie Stevens

\end{enhancedmcq}
\begin{enhancedmcq}{What is the name of the genetic disease that provided important insights into chromosome structure?}
\item Down syndrome
\item Klinefelter syndrome
\item Turner syndrome

\end{enhancedmcq}
\begin{enhancedmcq}{Which microscope type revolutionized cell biology by allowing scientists to see structures at the molecular level?}
\item Electron microscope
\item Confocal microscope
\item Phase contrast microscope

\end{enhancedmcq}
\begin{enhancedmcq}{What unique feature do naked mole‑rats have that makes them valuable in cancer research?}
\item Cancer resistance
\item Regenerative abilities
\item Extended lifespan

\end{enhancedmcq}
\begin{enhancedmcq}{Who proposed the endosymbiotic theory explaining the origin of mitochondria and chloroplasts?}
\item Lynn Margulis
\item Ernst Haeckel
\item Theodor Schwann

\end{enhancedmcq}
\begin{enhancedmcq}{What biological phenomenon was discovered during research on bacterial resistance to viruses?}
\item CRISPR‑Cas9
\item RNA interference
\item Restriction enzymes

\end{enhancedmcq}
\begin{enhancedmcq}{Which mammal can detect electrical signals from other animals?}
\item Platypus
\item Dolphin
\item Bat

\end{enhancedmcq}
\begin{enhancedmcq}{What is the term for the process by which certain bacteria transform into a different strain?}
\item Bacterial transformation
\item Conjugation
\item Transduction

\end{enhancedmcq}
\begin{enhancedmcq}{Which scientist created the first recombinant DNA molecule in 1972?}
\item Paul Berg
\item Herbert Boyer
\item Stanley Cohen

\end{enhancedmcq}
\begin{enhancedmcq}{What is the smallest known autonomous living organism?}
\item Mycoplasma genitalium
\item Nanoarchaeum equitans
\item Pelagibacter ubique

\end{enhancedmcq}
\begin{enhancedmcq}{Which evolutionary biologist developed the concept of punctuated equilibrium?}
\item Stephen Jay Gould
\item Richard Dawkins
\item E.O. Wilson

\end{enhancedmcq}
\begin{enhancedmcq}{What unique adaptation allows certain frogs to survive being completely frozen?}
\item Cryoprotectant production
\item Antifreeze proteins
\item Metabolic shutdown

\end{enhancedmcq}
\begin{enhancedmcq}{Which scientist's research on chromosomes led to the discovery of sex determination?}
\item Nettie Stevens
\item Rosalind Franklin
\item Rita Levi‑Montalcini

\end{enhancedmcq}
\begin{enhancedmcq}{What is the only known animal that never stops growing throughout its entire life?}
\item Greenland shark
\item Lobster
\item Bowhead whale

\end{enhancedmcq}
\begin{enhancedmcq}{Which plant has the largest genome of any studied organism?}
\item Paris japonica (Japanese canopy plant)
\item Sequoia sempervirens (Redwood)
\item Rafflesia arnoldii (Corpse flower)

\end{enhancedmcq}
\begin{enhancedmcq}{What revolutionary technique developed in 1983 allows scientists to make millions of copies of DNA?}
\item Polymerase Chain Reaction (PCR)
\item Gel electrophoresis
\item Southern blotting

\end{enhancedmcq}
\begin{enhancedmcq}{Which biologist coined the term "ecology" and defined it as the study of organisms and their environment?}
\item Ernst Haeckel
\item Alexander von Humboldt
\item Rachel Carson

\end{enhancedmcq}
\begin{enhancedmcq}{What is the name of the largest known virus, and what are its characteristics?}
\item Mimivirus
\item HIV
\item Ebola

\end{enhancedmcq}
\begin{enhancedmcq}{Which scientist first described the complete human circulatory system?}
\item William Harvey
\item Andreas Vesalius
\item Galen of Pergamon

\end{enhancedmcq}
\begin{enhancedmcq}{What is the only known biological structure that can catalyze its own synthesis?}
\item Ribozyme
\item Prion
\item Retrovirus

\end{enhancedmcq}
\begin{enhancedmcq}{Which biologist developed the concept of "selfish gene" to explain evolutionary behaviors?}
\item Richard Dawkins
\item E.O. Wilson
\item Stephen Jay Gould

\end{enhancedmcq}
\begin{enhancedmcq}{What is the term for specialized proteins that speed up biochemical reactions?}
\item Enzymes
\item Hormones
\item Antibodies

\end{enhancedmcq}
\begin{enhancedmcq}{Which marine microorganism produces most of Earth's oxygen?}
\item Prochlorococcus
\item Diatoms
\item Cyanobacteria

\end{enhancedmcq}
\begin{enhancedmcq}{What is the only known animal that can demonstrate self‑recognition in a mirror test?}
\item Great apes (including humans)
\item Elephants
\item Both a and b

\end{enhancedmcq}
\begin{enhancedmcq}{Which scientist developed the "one gene, one enzyme" hypothesis?}
\item George Beadle
\item Joshua Lederberg
\item Francis Crick

\end{enhancedmcq}
\begin{enhancedmcq}{What geological period saw the emergence of the first land plants?}
\item Ordovician
\item Silurian
\item Devonian

\end{enhancedmcq}
\begin{enhancedmcq}{Which female biologist's work with chimpanzees revolutionized our understanding of primate behavior?}
\item Jane Goodall
\item Dian Fossey
\item Biruté Galdikas

\end{enhancedmcq}
\begin{enhancedmcq}{What is the term for the recently discovered communication network between plants via underground fungal connections?}
\item Wood Wide Web
\item Mycorrhizal Network
\item Fungal Internet
\end{enhancedmcq}