\section{BASIC CONCEPTS OF BIOLOGY}

\textbf{Q1} Which Swedish botanist developed the binomial nomenclature system used to classify organisms?\par
\quad - a) Carl Linnaeus\par
\quad - b) Gregor Mendel\par
\quad - c) Charles Darwin\par

\textbf{Q2} What unique adaptation allows tardigrades (water bears) to survive in extreme environments, including the vacuum of space?\par
\quad - a) Cryptobiosis\par
\quad - b) Bioluminescence\par
\quad - c) Symbiosis\par

\textbf{Q3} Which scientist conducted the famous pea plant experiments that established the basic principles of heredity?\par
\quad - a) Gregor Mendel\par
\quad - b) Thomas Hunt Morgan\par
\quad - c) Barbara McClintock\par

\textbf{Q4} What is the only known organelle that contains its own separate DNA independent from the cell nucleus?\par
\quad - a) Mitochondria\par
\quad - b) Golgi apparatus\par
\quad - c) Endoplasmic reticulum\par

\textbf{Q5} Which female scientist's X‑ray crystallography work was crucial to discovering the DNA double helix structure?\par
\quad - a) Rosalind Franklin\par
\quad - b) Marie Curie\par
\quad - c) Barbara McClintock\par

\textbf{Q6} What is the term for animals that can regenerate entire body parts or organs?\par
\quad - a) Autotomous\par
\quad - b) Regenerative\par
\quad - c) Pluripotent\par

\textbf{Q7} Which neuroscientist was awarded the Nobel Prize for discovering split‑brain phenomenon?\par
\quad - a) Roger Sperry\par
\quad - b) Santiago Ramón y Cajal\par
\quad - c) Eric Kandel\par

\textbf{Q8} What animal has the largest brain‑to‑body‑weight ratio of any vertebrate?\par
\quad - a) Hummingbird\par
\quad - b) Dolphin\par
\quad - c) Human\par

\textbf{Q9} What is the name of the phenomenon where an isolated population develops unique adaptations?\par
\quad - a) Adaptive radiation\par
\quad - b) Genetic drift\par
\quad - c) Allopatric speciation\par

\textbf{Q10} Which scientist first described the cell nucleus in 1831?\par
\quad - a) Robert Brown\par
\quad - b) Anton van Leeuwenhoek\par
\quad - c) Robert Hooke\par

\textbf{Q11} What is the enzyme that repairs breaks in DNA during replication?\par
\quad - a) DNA ligase\par
\quad - b) DNA polymerase\par
\quad - c) Helicase\par

\textbf{Q12} Which scientist's observation of finches on the Galapagos Islands greatly influenced his theory of evolution?\par
\quad - a) Charles Darwin\par
\quad - b) Alfred Russel Wallace\par
\quad - c) Jean‑Baptiste Lamarck\par

\textbf{Q13} What unusual biological phenomenon allows some reptiles to reproduce without fertilization?\par
\quad - a) Parthenogenesis\par
\quad - b) Hermaphroditism\par
\quad - c) Polyembryony\par

\textbf{Q14} Which microbiologist developed the first effective vaccine against rabies?\par
\quad - a) Louis Pasteur\par
\quad - b) Robert Koch\par
\quad - c) Edward Jenner\par

\textbf{Q15} What is the process called when a species evolves to resemble another unrelated species?\par
\quad - a) Mimicry\par
\quad - b) Convergent evolution\par
\quad - c) Divergent evolution\par

\textbf{Q16} Which scientist is credited with discovering the concept of natural selection simultaneously with Darwin?\par
\quad - a) Alfred Russel Wallace\par
\quad - b) Thomas Huxley\par
\quad - c) Herbert Spencer\par

\textbf{Q17} What type of RNA carries amino acids to ribosomes during protein synthesis?\par
\quad - a) Transfer RNA (tRNA)\par
\quad - b) Messenger RNA (mRNA)\par
\quad - c) Ribosomal RNA (rRNA)\par

\textbf{Q18} Which animal's blood contains copper instead of iron, making it blue instead of red?\par
\quad - a) Bluefish\par
\quad - b) Octopus\par
\quad - c) Greyseal\par

\textbf{Q19} Which biochemist discovered how to determine the complete amino acid sequence of insulin?\par
\quad - a) Frederick Sanger\par
\quad - b) James Watson\par
\quad - c) Linus Pauling\par

\textbf{Q20} What is the term for plants that grow on other plants without being parasitic?\par
\quad - a) Epiphytes\par
\quad - b) Saprophytes\par
\quad - c) Xerophytes\par

\textbf{Q21} Which scientist first described the process of photosynthesis in plants?\par
\quad - a) Jan Ingenhousz\par
\quad - b) Joseph Priestley\par
\quad - c) Jean Senebier\par

\textbf{Q22} What type of cell division results in genetically identical daughter cells?\par
\quad - a) Mitosis\par
\quad - b) Meiosis\par
\quad - c) Binary fission\par

\textbf{Q23} Which woman received the Nobel Prize for her discovery of mobile genetic elements (jumping genes)?\par
\quad - a) Barbara McClintock\par
\quad - b) Lynn Margulis\par
\quad - c) Nettie Stevens\par

\textbf{Q24} What is the name of the genetic disease that provided important insights into chromosome structure?\par
\quad - a) Down syndrome\par
\quad - b) Klinefelter syndrome\par
\quad - c) Turner syndrome\par

\textbf{Q25} Which microscope type revolutionized cell biology by allowing scientists to see structures at the molecular level?\par
\quad - a) Electron microscope\par
\quad - b) Confocal microscope\par
\quad - c) Phase contrast microscope\par

\textbf{Q26} What unique feature do naked mole‑rats have that makes them valuable in cancer research?\par
\quad - a) Cancer resistance\par
\quad - b) Regenerative abilities\par
\quad - c) Extended lifespan\par

\textbf{Q27} Who proposed the endosymbiotic theory explaining the origin of mitochondria and chloroplasts?\par
\quad - a) Lynn Margulis\par
\quad - b) Ernst Haeckel\par
\quad - c) Theodor Schwann\par

\textbf{Q28} What biological phenomenon was discovered during research on bacterial resistance to viruses?\par
\quad - a) CRISPR‑Cas9\par
\quad - b) RNA interference\par
\quad - c) Restriction enzymes\par

\textbf{Q29} Which mammal can detect electrical signals from other animals?\par
\quad - a) Platypus\par
\quad - b) Dolphin\par
\quad - c) Bat\par

\textbf{Q30} What is the term for the process by which certain bacteria transform into a different strain?\par
\quad - a) Bacterial transformation\par
\quad - b) Conjugation\par
\quad - c) Transduction\par

\textbf{Q31} Which scientist created the first recombinant DNA molecule in 1972?\par
\quad - a) Paul Berg\par
\quad - b) Herbert Boyer\par
\quad - c) Stanley Cohen\par

\textbf{Q32} What is the smallest known autonomous living organism?\par
\quad - a) Mycoplasma genitalium\par
\quad - b) Nanoarchaeum equitans\par
\quad - c) Pelagibacter ubique\par

\textbf{Q33} Which evolutionary biologist developed the concept of punctuated equilibrium?\par
\quad - a) Stephen Jay Gould\par
\quad - b) Richard Dawkins\par
\quad - c) E.O. Wilson\par

\textbf{Q34} What unique adaptation allows certain frogs to survive being completely frozen?\par
\quad - a) Cryoprotectant production\par
\quad - b) Antifreeze proteins\par
\quad - c) Metabolic shutdown\par

\textbf{Q35} Which scientist's research on chromosomes led to the discovery of sex determination?\par
\quad - a) Nettie Stevens\par
\quad - b) Rosalind Franklin\par
\quad - c) Rita Levi‑Montalcini\par

\textbf{Q36} What is the only known animal that never stops growing throughout its entire life?\par
\quad - a) Greenland shark\par
\quad - b) Lobster\par
\quad - c) Bowhead whale\par

\textbf{Q37} Which plant has the largest genome of any studied organism?\par
\quad - a) Paris japonica (Japanese canopy plant)\par
\quad - b) Sequoia sempervirens (Redwood)\par
\quad - c) Rafflesia arnoldii (Corpse flower)\par

\textbf{Q38} What revolutionary technique developed in 1983 allows scientists to make millions of copies of DNA?\par
\quad - a) Polymerase Chain Reaction (PCR)\par
\quad - b) Gel electrophoresis\par
\quad - c) Southern blotting\par

\textbf{Q39} Which biologist coined the term "ecology" and defined it as the study of organisms and their environment?\par
\quad - a) Ernst Haeckel\par
\quad - b) Alexander von Humboldt\par
\quad - c) Rachel Carson\par

\textbf{Q40} What is the name of the largest known virus, and what are its characteristics?\par
\quad - a) Mimivirus\par
\quad - b) HIV\par
\quad - c) Ebola\par

\textbf{Q41} Which scientist first described the complete human circulatory system?\par
\quad - a) William Harvey\par
\quad - b) Andreas Vesalius\par
\quad - c) Galen of Pergamon\par

\textbf{Q42} What is the only known biological structure that can catalyze its own synthesis?\par
\quad - a) Ribozyme\par
\quad - b) Prion\par
\quad - c) Retrovirus\par

\textbf{Q43} Which biologist developed the concept of "selfish gene" to explain evolutionary behaviors?\par
\quad - a) Richard Dawkins\par
\quad - b) E.O. Wilson\par
\quad - c) Stephen Jay Gould\par

\textbf{Q44} What is the term for specialized proteins that speed up biochemical reactions?\par
\quad - a) Enzymes\par
\quad - b) Hormones\par
\quad - c) Antibodies\par

\textbf{Q45} Which marine microorganism produces most of Earth's oxygen?\par
\quad - a) Prochlorococcus\par
\quad - b) Diatoms\par
\quad - c) Cyanobacteria\par

\textbf{Q46} What is the only known animal that can demonstrate self‑recognition in a mirror test?\par
\quad - a) Great apes (including humans)\par
\quad - b) Elephants\par
\quad - c) Both a and b\par

\textbf{Q47} Which scientist developed the "one gene, one enzyme" hypothesis?\par
\quad - a) George Beadle\par
\quad - b) Joshua Lederberg\par
\quad - c) Francis Crick\par

\textbf{Q48} What geological period saw the emergence of the first land plants?\par
\quad - a) Ordovician\par
\quad - b) Silurian\par
\quad - c) Devonian\par

\textbf{Q49} Which female biologist's work with chimpanzees revolutionized our understanding of primate behavior?\par
\quad - a) Jane Goodall\par
\quad - b) Dian Fossey\par
\quad - c) Biruté Galdikas\par

\textbf{Q50} What is the term for the recently discovered communication network between plants via underground fungal connections?\par
\quad - a) Wood Wide Web\par
\quad - b) Mycorrhizal Network\par
\quad - c) Fungal Internet\par